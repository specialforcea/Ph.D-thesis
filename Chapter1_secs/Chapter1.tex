%Chapter 1

\renewcommand{\thechapter}{1}

\chapter{Theoretical description of Bose-Einstein condensates}
In this chapter, we discuss the fundamental theoretical description of non-interacting Bose-Einstein condensates (BEC) and interacting Bose-Einstein condensates. 

The non-interacting BEC section (Sec. \ref{sec. bec1}) starts from the difference in the description of particles between classical mechanics and quantum mechanics. Indistinguishability and the exclusion principle of identical quantum particles lead to dramatically different statistical properties from classical particles. Next, I briefly discuss the exchange symmetry along with the reason why it is related to particles' spin. After the discussion of exchange symmetry, I derive Bose-Einstein statistics and discuss how it leads to non-interacting BECs.

In the many-body BEC section (Sec. \ref{sec MB}), I  first introduce the second quantization description of many-body systems. Next, by making the mean-field approximation, I derive the Gross-Pitaevskii equation (GPE) for scalar and spinor bosonic systems.
\section{Bose-Einstein statistics}
\subsection{Classical description of particles}
Classical mechanics originate from Newtonian mechanics and have different but equivalent formulations. It assumes the position and velocity of an object can be measured and kept track of at any time. The dynamics of an object or a system of objects can be sufficiently described by their positions and velocities. For example, in one of the equivalent formulations, Hamiltonian mechanics \cite{goldstein2002classical}, a classical system is described by a set of canonical coordinates $\vec{r} = (\vec{p},\vec{q})$. Here $\vec{p} = (p_1,p_2,\cdots,p_N)$  and $\vec{q}= (q_1,q_2,\cdots,q_N)$, they are indexed by the N-dimensional frame of reference of the system. Hamiltonian $\mathcal{H} = \mathcal{H}(\vec{p},\vec{q},t)$ is a function of canonical coordinates and corresponds to total energy of a system. The dynamics of the system is governed by equations
\begin{equation}\label{cla ham}
\begin{aligned}
    &\frac{d\vec{p}}{dt} = -\frac{\partial\mathcal{H}}{\partial\vec{q}}\\
   &\frac{d\vec{q}}{dt} = \frac{\partial\mathcal{H}}{\partial\vec{p}}
\end{aligned}
\end{equation}
Both $\vec{p}$ and $\vec{q}$ evolve with equation \ref{cla ham} deterministically and for a system of classical particles, each particle's trajectory can be traced during evolution, making them distinguishable from each other.

In classical statistical mechanics, the Maxwell-Boltzmann statistics describes the distribution of non-interacting particles in thermal equilibrium over energy states. The ensemble averaged number of particles in energy state $\epsilon_i$ is
\begin{equation}
    n_i = \frac{g_i}{\exp{(\epsilon_i-\mu)/k_BT}}.
\end{equation}
Here, $g_i$ is the degeneracy of energy state $\epsilon_i$, $\mu$ is the chemical potential which can be obtained from the conservation of particle number. $k_B$ is the Boltzmann's constant and $T$ is temperature.  
\subsection{Quantum description of particles}
Quantum mechanics, formed by six postulates, is very powerful in explaining modern experiments in atomic, molecular, optical and condensed matter physics and so on. It also introduces counter-intuitive concepts such as photon, matter wave and Pauli's exclusion principle. These concepts won't be well understood without a good sense of the postulates which are listed in appendix \ref{appendix:QMP}.  

As the first postulate states, a particle is described by its quantum state $\ket{\Psi(t)}$. $\ket{\Psi(t)}$ contains all the information about the particle and can be represented by projecting it in any complete basis. For example, $\Psi(\mathbf{r},t) = \bra{\mathbf{r}}\ket{\Psi(t)}$ is defined as the spatial wave function of the particle. Since $\ket{\mathbf{r}}s$ form a complete basis, wave function $\Psi(\mathbf{r},t)$ is sufficient to represent state $\ket{\Psi(t)}$. Similarly, projecting $\Psi(t)$ in k space, we can define wave function in momentum space $\Psi(\mathbf{p},t) = \bra{\mathbf{p}}\ket{\Psi(t)}$. As postulate four states, for a particle in state $\Psi(t)$, without measurement, the only information available to us is the probability of the particle being in position $\mathbf{r}$ and momentum $\mathbf{p}$, it is not possible to know the position and momentum of the particle without measurement. Moreover, as postulate five states, immediately after the measurement of an observable, the state collapse to one of its eigenstates, meaning after a measurement, the particle's state will be changed. 
\subsection{Distinguishability}
For a group of identical particles, their spatial state can overlap at any time, and when they do, the lack of information and inability to do measurements without changing their states make it in principle impossible to track the particles.  And thus, in quantum mechanics, identical particles are indistinguishable. This indistinguishability has a fundamental effect on the statistics of particles.

\subsection{Exchange symmetry of identical particles}

For a group of identical particles, the many-body state describing the system should not have any measurable difference exchanging any two particles of them. In quantum mechanics, a state is determined up to a phase factor. For a state $\ket{\Psi(t)}$, $e^{i\theta}\ket{\Psi(t)}$ does not have any measurable difference for any $\theta \in \mathbb{R}$. So for a two-particle state $\ket{\psi_1\psi_2}$, exchange the two particles, in the most general case, the two-particle state should become $e^{i\theta}\ket{\psi_2\psi_1}$. People may argue that exchanging particles twice the state should return, $e^{i2\theta} = 1$, but it is not rigorous. It was introduced as symmetry postulate that $e^{i\theta}$ can only take the value of $+1$ or $-1$, meaning the two-particle state can only be symmetric or anti-symmetric. It has deep consequences in particle statistics and is in good agreement with experimental facts. However, its theoretical justification remains unclear. In 1977, J.M.Leinass and J.Myrheim introduced in their paper \cite{leinaas1977theory} a quantization formalism, in which the restriction on wave function to be either symmetric or antisymmetric appears in a natural way, without having to add any additional constraints. However, this is true only when space is at least three-dimensional. In one or two dimensions, $e^{i\theta}$ can take other values. Frank Wilczek invented the term "anyon" to describe this kind of particle and they are important in understanding the fractional quantum Hall effect.   


\subsection{Bosons, Fermions and spin statistics}
In three dimensional space, particles' wave function can only be symmetric or anti-symmetric. The symmetry of the wave function is determined by the particles' spin. A particle is called Boson if it has an integer spin and Fermion if it has a half-integer spin. In consequence, the wave functions of Bosons and Fermions are symmetric and anti-symmetric, respectively.  The spin statistics theorem is first formulated in 1939 by Markus Fierz\cite{fierz1939relativistische} and rederived in a more systematic way in 1940 by Wolfgang Pauli\cite{pauli1940connection}. A more conceptual argument was provided in 1950 by Julian Schwinger. It is fascinating and very non-intuitive how particles' spin determines their exchange symmetry. As Feynman commented in his book \textit{Feynman Lectures on Physics}\cite{feynman2011feynman}: \textit{An explanation has been worked out by Pauli from complicated arguments of QFT and relativity. But we have not found a way of reproducing his arguments on an elementary level. This probably means that we do not have a complete understanding of the fundamental principle involved.}
  
\subsection{Bose-Einstein statistics and non-interacting BEC}
Quantum statistics differs from classical statistics in two aspects, indistinguishability and exchange symmetry. 

For a two-particle wave function $\psi(\mathbf{r}_1,\mathbf{r}_2)$, if the two particles are Fermions, they can not occupy the same state. $\psi(\mathbf{r}_1,\mathbf{r}_2) = -\psi(\mathbf{r}_2,\mathbf{r}_1)$, if $\psi(\mathbf{r}_1,\mathbf{r}_2) = \psi(\mathbf{r}_2,\mathbf{r}_1)$, $\psi(\mathbf{r}_1,\mathbf{r}_2) = 0$. For Bosons, this is not the case. Any number of Bosons can occupy the same state and the system has the lowest energy when all the Bosons are in the ground state.

The Bose-Einstein distribution can be derived from the principle of maximum entropy. A Boson system with $N$ particles are distributed to states $\{\ket{\epsilon_i}\}$, each state $\ket{\epsilon_i}$ has degeneracy $g_i$ and it is occupied by $n_i$ particles. The number of micro-state is
\begin{equation}
    \Omega(\{n_i\}) = \prod_{i} \frac{(n_i+g_i-1)!}{n_i!(g_i-1)!}.
\end{equation}
Maximizing $\Omega(\{n_i\})$ under the constraints $\sum_{i}n_i = N$ and $\sum_{i}n_i\epsilon_i = U$, we can derive Bose-Einstein distribution
\begin{equation}
    n_i = \frac{1}{\exp{-\alpha-\beta\epsilon_i}-1},
\end{equation}
$\alpha$ and $\beta$ are the Lagrange multipliers. From constraints $\sum_{i}n_i = N$ and $\sum_{i}n_i\epsilon_i = U$, it can be determined $\alpha$ is chemical potential $\mu$ over $k_BT$ and $\beta$ is $-1/k_BT$ where $k_B$ is Boltzmann constant and $T$ is temperature.

To find out how does this distribution lead to Bose-Einstein condensates, we need to revisit the constraints. For a system of particles in a box of volume V. The energy density of states is
\begin{equation}
    g(\epsilon) = \frac{V}{4\pi^2}\big(\frac{2m}{\hbar}\big)^{3/2}\sqrt{\epsilon}    
\end{equation}
Replacing the sum in $\sum_{i}n_i = N$ with integral, we get
\begin{equation}\label{N}
    \frac{V}{4\pi^2}\big(\frac{2m}{\hbar}\big)^{3/2}\int_0^\infty\frac{\sqrt{\epsilon}}{\exp{(\epsilon-\mu)/k_BT}-1}d\epsilon = N
\end{equation}
Denote the integral as $I(\mu)$
\begin{equation}
    I(\mu) = \int_0^\infty\frac{\sqrt{\epsilon}}{\exp{(\epsilon-\mu)/k_BT}-1}d\epsilon
\end{equation}
For Eq.~(\ref{N}) to hold, $I(\mu)$ should be a constant, and $\mu$ should be a function of $T$. As $T$ decreases, $\mu$ should increase to keep $I(\mu)$ constant. But $\mu$ can not go positive because in that case, for states $\epsilon_i-\mu < 0$, $n_i$ will be negative. So as $T$ decreases to some critical temperature $T_c$, $\mu$ increases to 0 to keep $I(\mu)$ constant. As $T$ continue decreasing, $\mu$ can not increase anymore and we have
\begin{equation}
    \frac{V}{4\pi^2}\big(\frac{2m}{\hbar}\big)^{3/2}\int_0^\infty\frac{\sqrt{\epsilon}}{\exp{(\epsilon-\mu)/k_BT}-1}d\epsilon < N
\end{equation}
Where are the missing particles? The answer is they are condensed in the ground state with zero energy. If we look at the density of state $g(\epsilon)$, it is zero for the ground state. When $T$ is high, the number of particles that are in the ground state is negligible, so the integral counts all the particles and equals $N$. When $T<T_c$, $\mu$ is zero, a significant amount of particles start to occupy the ground state, and the missed particles will cause the loss of atoms counts. 

The critical temperature can be calculated from Eq.~(\ref{N}) with $\mu=0$.
\begin{equation}
    T_c = \frac{2\pi\hbar^2}{mk_B}\Big(\frac{N}{\eta(3/2)V}\Big)^{2/3}
\end{equation}
where $\eta(x)$ is Riemann zeta function. And it can be shown the fraction of particles in the ground state is a function of $T$ and $T_c$
\begin{equation}
    \frac{N_0}{N} = 1-\Big(\frac{T}{T_c}\Big)^{3/2}.
\end{equation}






\section{Manybody BEC system}\label{sec MB}
\subsection{Second quantization}
An interacting system can be described by the second quantization Hamiltonian. In the second quantization framework, particles in a system are described by creation and annihilation operators on the basis of many-body Fock states. Creation operators $\hat{a}_k^\dag$ creates a particle in the Fock state $\ket{n_1,\dots , n_k,\dots}$ while annihilation operators $\hat{a}_k$ annihilates a particle in the Fock state $\ket{n_1,\dots , n_k,\dots}$.  
\begin{equation}
\begin{aligned}
    &\hat{a}_k^\dag\ket{n_1,\dots , n_k,\dots} = \sqrt{n_k+1}\ket{n_1,\dots,n_k+1,\dots}\\
    &\hat{a}_k\ket{n_1,\dots,n_k,\dots} = \sqrt{n_k}\ket{n_1,\dots,n_k-1,\dots}
\end{aligned}
\end{equation}
With a change of basis, the spatial creation and annihilation operators can be defined as
\begin{equation}
    \begin{aligned}
    &\hat{\psi}(\Vec{r}) = \sum_k \hat{a}_k \braket{\Vec{r}}{\Vec{k}} = \sum_k \hat{a}_k \frac{\exp{i\Vec{k}\cdot\Vec{r}}}{\sqrt{V}}\\
    &\hat{\psi}^\dag(\Vec{r}) = \sum_k \hat{a}^\dag_k \braket{\Vec{r}}{\Vec{k}} = \sum_k \hat{a}_k^\dag \frac{\exp{-i\Vec{k}\cdot\Vec{r}}}{\sqrt{V}}.\\
    \end{aligned}
\end{equation}
$\hat{\psi}^\dag(\Vec{r})$ and $\hat{\psi}(\Vec{r})$ creates and annihilates a particle at spatial coordinate $\Vec{r}$. Operator $\hat{N}_k = \hat{a}_k^\dag\hat{a}_k$ counts the number of particles in the momentum state $\ket{k}$. The two-particle operator is defined as 
\begin{equation}
    \begin{aligned}
    \hat{V}_{int} &= \frac{1}{2} \sum_{i\neq j}\hat{N}_i\hat{N}_j V_{ij} + \frac{1}{2}\sum_i \hat{N}_i(\hat{N}_i-1)V_{ii}\\
    & = \frac{1}{2}\sum_{i, j}(\hat{N}_i\hat{N}_j - \hat{N}_i\delta_{ij})V_{ij}
    \end{aligned}
\end{equation}
where $V_{ij}$ is the interaction between particles in states $\ket{i}$ and $\ket{j}$. From the orthogonality of Fock states and Pauli's exclusion principle, we can derive the commutators of creation and annihilation operators for both Bosons and Fermions.
For Bosons,
\begin{equation}
    \begin{aligned}
    &[\hat{a}_i, \hat{a}_j^\dag] = \delta_{ij}\\
    &[\hat{a}_i, \hat{a}_j] = [\hat{a}_i^\dag, \hat{a}_j^\dag] = 0
    \end{aligned}
\end{equation}
where $[A,B] = AB - BA$ is the commutator of operators $A$ and $B$.
For Fermions,
\begin{equation}
    \begin{aligned}
    &\{\hat{a}_i, \hat{a}_j^\dag\} = \delta_{ij}\\
    &\{\hat{a}_i, \hat{a}_j\} = \{\hat{a}_i^\dag, \hat{a}_j^\dag\} = 0
    \end{aligned}
\end{equation}
where $\{A,B\} = AB + BA$ is the anti-commutator of operators $A$ and $B$. With the change of basis, we can also obtain the commutation and anti-commutation relations of field operators $\hat{\psi}(\Vec{r})$ and $\hat{\psi}^\dag(\Vec{r})$.
For Bosons,
\begin{equation}
    \begin{aligned}
    &[\hat{\psi}(\Vec{r}~'), \hat{\psi}^\dag(\Vec{r}~'')] = \delta^3(\Vec{r}~'-\Vec{r}~'')\\
    &[\hat{\psi}(\Vec{r}~'), \hat{\psi}(\Vec{r}~'')] = [\hat{\psi}^\dag(\Vec{r}~'), \hat{\psi}^\dag(\Vec{r}~'')] = 0.
    \end{aligned}
\end{equation}
For Fermions,
\begin{equation}
    \begin{aligned}
    &\{\hat{\psi}(\Vec{r}~'), \hat{\psi}^\dag(\Vec{r}~'')\} = \delta^3(\Vec{r}~'-\Vec{r}~'')\\
    &\{\hat{\psi}(\Vec{r}~'), \hat{\psi}(\Vec{r}~'')\} = \{\hat{\psi}^\dag(\Vec{r}~'), \hat{\psi}^\dag(\Vec{r}~'')\} = 0.
    \end{aligned}
\end{equation}
With the commutators, we can rewrite the interaction operator $\hat{V}$,
\begin{equation}
    \hat{V}_{int} = \frac{1}{2}\sum_{i,j}\hat{a}_i^\dag\hat{a}_j^\dag V_{ij} \hat{a}_j\hat{a}_i,
\end{equation}
which is valid for both Bosons and Fermions.

In general, the Hamiltonian of a system is
\begin{equation}
    \hat{H} = \sum_i \frac{\hat{\textbf{p}}_i^2}{2m} + \hat{V}(\hat{\textbf{r}}_i) + \hat{V}_{int},
\end{equation}
in the representation of field operators, it takes the form
\begin{align}
    \hat{H} = &\int d\textbf{r} \hat{\Psi}^\dag(\textbf{r})\Bigg[-\frac{\hbar^2}{2m}\grad^2 + \hat{V}(\textbf{r})\bigg]\hat{\Psi}(\textbf{r})\\
    & + \frac{1}{2}\int d\textbf{r}'d\textbf{r}''d\textbf{r}'''d\textbf{r}''''\hat{\Psi}^\dag(\textbf{r}')\hat{\Psi}^\dag(\textbf{r}'')\matrixel{\textbf{r}'\textbf{r}''}{\hat{V}_{int}}{\textbf{r}'''\textbf{r}''''}\hat{\Psi}(\textbf{r}''')\hat{\Psi}(\textbf{r}'''')
\end{align}



\subsection{GPE of scalar BEC}
For dilute and cold gases, it is proper to approximate the interactions between atoms with two-body collisions. At low energy, the two-body collision interactions can be represented by the s-wave pseudopotential which is characterized by s-wave scattering length.
\begin{equation}
    \hat{V}_{int} = g\delta(\textbf{r}' - \textbf{r}''), 
\end{equation}
the constant $g$ is a function of scattering length $a$,
\begin{equation}
    g = \frac{4\pi\hbar^2a}{m}.
\end{equation}

The Hamiltonian of the cold gases system takes the form
\begin{align}
    \hat{H} = &\int d\textbf{r} \hat{\Psi}^\dag(\textbf{r})\left[-\frac{\hbar^2}{2m}\grad^2 + \hat{V}(\textbf{r})\right]\hat{\Psi}(\textbf{r})\\ \nonumber
    & + \frac{g}{2}\int d\textbf{r}\hat{\Psi}^\dag(\textbf{r})\hat{\Psi}^\dag(\textbf{r})\hat{\Psi}(\textbf{r})\hat{\Psi}(\textbf{r}),
\end{align}
and the dynamics of the field operators obey Heisenberg equation
\begin{align}
    i\hbar\frac{\partial}{\partial t}\hat{\Psi}(\textbf{r},t) &= \left[ \hat{\Psi}, \hat{H}\right]\\\nonumber
    &=\left[-\frac{\hbar^2}{2m}\grad^2 + \hat{V}(\textbf{r}) + g\hat{\Psi}^\dag(\textbf{r},t)\hat{\Psi}(\textbf{r},t) \right] \hat{\Psi}(\textbf{r},t)
\end{align}

Bogoliubov formulated the mean-field approach \cite{Bogolyubov:1947zz} to solving the cold dilute gases system by expanding the field operators to the first order,
\begin{equation}
    \hat{\Psi}(\textbf{r},t) = \Psi(\textbf{r},t) + \delta \hat{\Psi}(\textbf{r},t)
\end{equation}
$\Psi(\textbf{r},t)$ is the mean-field, the expectation value of field operator $\hat{\Psi}(\textbf{r},t)$, $\Psi(\textbf{r},t) = \expval{\hat{\Psi}(\textbf{r},t)}$. And $\delta \hat{\Psi}(\textbf{r},t)$ is the excitation term which describes the variation of the field operator around its expectation value. $\Psi(\textbf{r},t)$ is a classical field and is often called the wave function of the condensate. When excitation is small and can be neglected, we arrive at the dynamics of the classical field
\begin{equation}
    i\hbar\frac{\partial}{\partial t}\Psi(\textbf{r},t) =\left[-\frac{\hbar^2}{2m}\grad^2 + V(\textbf{r}) + g|\Psi(\textbf{r},t)|^2 \right]\Psi(\textbf{r},t)
\end{equation}

This equation is known as the Gross-Pitaevskii equation (GPE) which is derived by Gross \cite{gross1961structure} and Pitaevskii \cite{pitaevskii1961vortex} independently.

To obtain the ground state of the condensate, we can write the function $\Psi(\textbf{r},t)$ as $\Psi(\textbf{r},t) = \psi(\textbf{r})e^{-i\mu t/\hbar}$. Then the GPE becomes
\begin{equation}
    \mu \psi(\textbf{r}) =\left[-\frac{\hbar^2}{2m}\grad^2 + V(\textbf{r}) + g|\psi(\textbf{r})|^2 \right]\psi(\textbf{r})
\end{equation}
$\mu$ is chemical potential and is subject to the conservation of particle numbers
\begin{equation}
    \int d\textbf{r} |\psi(\textbf{r})|^2 = N
\end{equation}

At low temperature, $T \ll T_c$, the chemical potential of the system is dominated by the interaction energy which is much larger than the kinetic energy, since the majority of atoms are in their ground states $\ket{k=0}$. By making the Thomas-Fermi approximation, we can neglect the kinetic energy term and the ground state takes a simple form
\begin{equation}
     \psi(\textbf{r}) = \sqrt{\frac{\mu - V(\textbf{r})}{g}}
\end{equation}
This is the Thomas-Fermi wave function and given external potential $V(\textbf{r})$, chemical potential $\mu$ can be calculated under the constraint of atom-number conservation. 




\subsection{GPE of spinor BEC}
When we add the spin degree of freedom, we need to consider the spin-dependent potentials and the spin-spin interactions in the two-particle scattering processes. The two-body interaction takes the form
\begin{equation}
    \hat{V}_{int}(\textbf{r}_1,\textbf{r}_2) = (c_0 + c_2\Vec{F}_1\cdot\Vec{F}_2)\delta(\textbf{r}_1-\textbf{r}_2)
\end{equation}
The total angular momentum is conserved in the processes of two spinor collisions. And for two spin-1 particles, their total angular momentum $F$ can be 0 or 2, corresponding to scattering channel 0 and channel 2. The sign of the coefficient $c_2$ determines whether the system is ferromagnetic (FM) or anti-ferromagnetic (AFM).  
The s-wave pseudo-potential for both channels is  
\begin{equation}
    V_{int}^F({\bf r},{\bf r}') = g_F\delta({\bf r},{\bf r}')
\end{equation}
where
\begin{equation}
    g_F = \frac{4\pi\hbar^2}{M}a_F
\end{equation}
and $F \in \{0,2\}$, $a_F$ is the scattering length. To calculate the coefficients $c_0$ and $c_2$, we need to project the two spin states to the total spin states.

In second quantization representation, the interaction of a spin-1 system \cite{kawaguchi2012spinor} is
\begin{equation}
    \hat{V}_{int} = \frac{1}{2} \sum_{m_1,m_2,m_1',m_2'}\int d\textbf{r}~ C_{m_1,m_2}^{m_1',m_2'} \hat{\Psi}^\dag_{m_1}(\textbf{r})\hat{\Psi}^\dag_{m_2}(\textbf{r})
    \hat{\Psi}_{m_1'}(\textbf{r})\hat{\Psi}_{m_2'}(\textbf{r})
\end{equation}
where
\begin{equation}
    C_{m_1,m_2}^{m_1',m_2'} = \frac{4\pi\hbar^2}{M} \sum_{F=0,2}a_F\matrixel{m_1;m_2}{\hat{P}_F}{m_1';m_2'}.
\end{equation}
Here the operator 
\begin{equation}
    \hat{P}_F = \sum_{m_F=-F}^F \ketbra{F,m_F}{F,m_F}
\end{equation}
projects the two-body angular momentum states onto the total angular momentum basis.

Back to the two-body angular momentum basis, the interaction can be written in the form
\begin{equation}
    \hat{V}_{int} = \frac{1}{2} \sum_{m_1,m_2}\int d\textbf{r}~  \hat{\Psi}^\dag_{m_1}(\textbf{r})\hat{\Psi}^\dag_{m_2}(\textbf{r})
    \hat{\Psi}_{m_1}(\textbf{r})\hat{\Psi}_{m_2}(\textbf{r})(c_0 + c_2\Vec{F}_{m_1}\cdot\Vec{F}_{m_2}).
\end{equation}
Here,
\begin{equation}
    c_0 = \frac{4\pi\hbar^2}{M}\frac{g_0+2g_2}{3},~ {\it and}~ c_2 = \frac{4\pi\hbar^2}{M}\frac{g_2 - g_0}{3}. 
\end{equation}

The Hamiltonian of the system is
\begin{equation}
    \hat{H} = \sum_{m_1,m_2} \int d\textbf{r}~ \hat{\Psi}^\dag_{m_1}(\textbf{r})\left[-\frac{\hbar^2}{2m}\grad^2 \delta_{m_1,m_2}+ \hat{U}_{m_1,m_2}(\textbf{r})\right]\hat{\Psi}_{m_2}(\textbf{r}) + \hat{V}_{int}.
\end{equation}
$\hat{U}_{m_1,m_2}(\textbf{r})$ can be any spin dependent potential. For example, Raman coupling is widely used in a BEC system as one way to realize the spin-orbit coupling. Intuitively, in a two-photon process, a particle in the spin state $\ket{m_F}$ absorbs a photon with momentum $k_{\rm R}$ and emits a photon with momentum $-k_{\rm R}$. The spin state jumps to $\ket{m_F + 1}$ and acquires momentum $2k_{\rm R}$. In the reverse process, it jumps from $\ket{m_F + 1}$ to $\ket{m_F}$ and acquires momentum $-2k_{\rm R}$. States $\ket{k + 2\kr,m_F+1}$ and $\ket{k - 2\kr,m_F}$ are coupled.

From Heisenberg equation and the mean field approximation, we can derive the spin-1 GPE
\begin{align} \label{spinor gpe}
    i\hbar\partial_t\psi_1 = & -\frac{\hbar^2}{2m}\nabla^2\psi_1  + \sum_{m_F}\hat{U}_{m_F,1}\psi_{m_F} + c_0n\psi_1\nonumber + c_2(n_1 + n_0 -n_{-1})\psi_1 + c_2\psi^\ast_{-1}\psi_0\psi_0\nonumber \\  
        i\hbar\partial_t\psi_0 = &-\frac{\hbar^2}{2m}\nabla^2\psi_0 + \sum_{m_F}\hat{U}_{m_F,0}\psi_{m_F} + c_0n\psi_0 + c_2(n_1 + n_{-1})\psi_0 +2c_2\psi^\ast_{0}\psi_1\psi_{-1}\nonumber\\
    i\hbar\partial_t\psi_{-1} = &-\frac{\hbar^2}{2m}\nabla^2\psi_{-1} + \sum_{m_F}\hat{U}_{m_F,-1}\psi_{m_F} + c_0n\psi_{-1}\nonumber +c_2(n_{-1} + n_0 -n_1)\psi_{-1} + c_2\psi^\ast_1\psi_0\psi_0\nonumber
\end{align}





