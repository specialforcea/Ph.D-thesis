\subsection{GPE of spinor BEC}
When we add the spin degree of freedom, we need to consider the spin-dependent potentials and the spin-spin interactions in the two-particle scattering processes. The two-body interaction takes the form
\begin{equation}
    \hat{V}_{int}(\textbf{r}_1,\textbf{r}_2) = (c_0 + c_2\Vec{F}_1\cdot\Vec{F}_2)\delta(\textbf{r}_1-\textbf{r}_2)
\end{equation}
The total angular momentum is conserved in the processes of two spinor collisions. And for two spin-1 particles, their total angular momentum $F$ can be 0 or 2, corresponding to scattering channel 0 and channel 2. The sign of the coefficient $c_2$ determines whether the system is ferromagnetic (FM) or anti-ferromagnetic (AFM).  
The s-wave pseudo-potential for both channels is  
\begin{equation}
    V_{int}^F({\bf r},{\bf r}') = g_F\delta({\bf r},{\bf r}')
\end{equation}
where
\begin{equation}
    g_F = \frac{4\pi\hbar^2}{M}a_F
\end{equation}
and $F \in \{0,2\}$, $a_F$ is the scattering length. To calculate the coefficients $c_0$ and $c_2$, we need to project the two spin states to the total spin states.

In second quantization representation, the interaction of a spin-1 system \cite{kawaguchi2012spinor} is
\begin{equation}
    \hat{V}_{int} = \frac{1}{2} \sum_{m_1,m_2,m_1',m_2'}\int d\textbf{r}~ C_{m_1,m_2}^{m_1',m_2'} \hat{\Psi}^\dag_{m_1}(\textbf{r})\hat{\Psi}^\dag_{m_2}(\textbf{r})
    \hat{\Psi}_{m_1'}(\textbf{r})\hat{\Psi}_{m_2'}(\textbf{r})
\end{equation}
where
\begin{equation}
    C_{m_1,m_2}^{m_1',m_2'} = \frac{4\pi\hbar^2}{M} \sum_{F=0,2}a_F\matrixel{m_1;m_2}{\hat{P}_F}{m_1';m_2'}.
\end{equation}
Here the operator 
\begin{equation}
    \hat{P}_F = \sum_{m_F=-F}^F \ketbra{F,m_F}{F,m_F}
\end{equation}
projects the two-body angular momentum states onto the total angular momentum basis.

Back to the two-body angular momentum basis, the interaction can be written in the form
\begin{equation}
    \hat{V}_{int} = \frac{1}{2} \sum_{m_1,m_2}\int d\textbf{r}~  \hat{\Psi}^\dag_{m_1}(\textbf{r})\hat{\Psi}^\dag_{m_2}(\textbf{r})
    \hat{\Psi}_{m_1}(\textbf{r})\hat{\Psi}_{m_2}(\textbf{r})(c_0 + c_2\Vec{F}_{m_1}\cdot\Vec{F}_{m_2}).
\end{equation}
Here,
\begin{equation}
    c_0 = \frac{4\pi\hbar^2}{M}\frac{g_0+2g_2}{3},~ {\it and}~ c_2 = \frac{4\pi\hbar^2}{M}\frac{g_2 - g_0}{3}. 
\end{equation}

The Hamiltonian of the system is
\begin{equation}
    \hat{H} = \sum_{m_1,m_2} \int d\textbf{r}~ \hat{\Psi}^\dag_{m_1}(\textbf{r})\left[-\frac{\hbar^2}{2m}\grad^2 \delta_{m_1,m_2}+ \hat{U}_{m_1,m_2}(\textbf{r})\right]\hat{\Psi}_{m_2}(\textbf{r}) + \hat{V}_{int}.
\end{equation}
$\hat{U}_{m_1,m_2}(\textbf{r})$ can be any spin dependent potential. For example, Raman coupling is widely used in a BEC system as one way to realize the spin-orbit coupling. Intuitively, in a two-photon process, a particle in the spin state $\ket{m_F}$ absorbs a photon with momentum $k_{\rm R}$ and emits a photon with momentum $-k_{\rm R}$. The spin state jumps to $\ket{m_F + 1}$ and acquires momentum $2k_{\rm R}$. In the reverse process, it jumps from $\ket{m_F + 1}$ to $\ket{m_F}$ and acquires momentum $-2k_{\rm R}$. States $\ket{k + 2\kr,m_F+1}$ and $\ket{k - 2\kr,m_F}$ are coupled.

From Heisenberg equation and the mean field approximation, we can derive the spin-1 GPE
\begin{align} \label{spinor gpe}
    i\hbar\partial_t\psi_1 = & -\frac{\hbar^2}{2m}\nabla^2\psi_1  + \sum_{m_F}\hat{U}_{m_F,1}\psi_{m_F} + c_0n\psi_1\nonumber + c_2(n_1 + n_0 -n_{-1})\psi_1 + c_2\psi^\ast_{-1}\psi_0\psi_0\nonumber \\  
        i\hbar\partial_t\psi_0 = &-\frac{\hbar^2}{2m}\nabla^2\psi_0 + \sum_{m_F}\hat{U}_{m_F,0}\psi_{m_F} + c_0n\psi_0 + c_2(n_1 + n_{-1})\psi_0 +2c_2\psi^\ast_{0}\psi_1\psi_{-1}\nonumber\\
    i\hbar\partial_t\psi_{-1} = &-\frac{\hbar^2}{2m}\nabla^2\psi_{-1} + \sum_{m_F}\hat{U}_{m_F,-1}\psi_{m_F} + c_0n\psi_{-1}\nonumber +c_2(n_{-1} + n_0 -n_1)\psi_{-1} + c_2\psi^\ast_1\psi_0\psi_0\nonumber
\end{align}


