\subsection{Exchange symmetry of identical particles}

For a group of identical particles, the many-body state describing the system should not have any measurable difference exchanging any two particles of them. In quantum mechanics, a state is determined up to a phase factor. For a state $\ket{\Psi(t)}$, $e^{i\theta}\ket{\Psi(t)}$ does not have any measurable difference for any $\theta \in \mathbb{R}$. So for a two-particle state $\ket{\psi_1\psi_2}$, exchange the two particles, in the most general case, the two-particle state should become $e^{i\theta}\ket{\psi_2\psi_1}$. People may argue that exchanging particles twice the state should return, $e^{i2\theta} = 1$, but it is not rigorous. It was introduced as symmetry postulate that $e^{i\theta}$ can only take the value of $+1$ or $-1$, meaning the two-particle state can only be symmetric or anti-symmetric. It has deep consequences in particle statistics and is in good agreement with experimental facts. However, its theoretical justification remains unclear. In 1977, J.M.Leinass and J.Myrheim introduced in their paper \cite{leinaas1977theory} a quantization formalism, in which the restriction on wave function to be either symmetric or antisymmetric appears in a natural way, without having to add any additional constraints. However, this is true only when space is at least three-dimensional. In one or two dimensions, $e^{i\theta}$ can take other values. Frank Wilczek invented the term "anyon" to describe this kind of particle and they are important in understanding the fractional quantum Hall effect.   

