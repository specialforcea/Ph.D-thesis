\subsection{Bosons, Fermions and spin statistics}
In three dimensional space, particles' wave function can only be symmetric or anti-symmetric. The symmetry of the wave function is determined by the particles' spin. A particle is called Boson if it has integer spin and Fermion if it has half-integer spin. In consequence, the wave function of Bosons and Fermions is symmetric and anti-symmetric, respectively.  The spin statistics theorem is first formulated in 1939 by Markus Fierz\cite{fierz1939relativistische} and rederived in a more systematic way in 1940 by Wolfgang Pauli\cite{pauli1940connection}. A more conceptual argument was provided in 1950 by Julian Schwinger. It is fascinating and very nonintuitive how particles' spin determines their exchange symmetry. As Feynman commented in his book \textit{Feynman Lectures on Physics}\cite{feynman2011feynman}: \textit{An explanation has been worked out by Pauli from complicated arguments of QFT and relativity. But we haven not found a way of reproducing his arguments on an elementary level. This probably means that we do not have a complete understanding of the fundamental principle involved.}
  