\subsection{Quantum description of particles}
Quantum mechanics, formed by six postulates, is very powerful in explaining modern experiments in atomic, molecular, optical and condensed matter physics and so on. It also introduces counter-intuitive concepts such as photon, matter wave and Pauli's exclusion principle. These concepts won't be well understood without a good sense of the postulates which are listed in appendix \ref{appendix:QMP}.  

As the first postulate states, a particle is described by its quantum state $\ket{\Psi(t)}$. $\ket{\Psi(t)}$ contains all the information about the particle and can be represented by projecting it in any complete basis. For example, $\Psi(\mathbf{r},t) = \bra{\mathbf{r}}\ket{\Psi(t)}$ is defined as the spatial wave function of the particle. Since $\ket{\mathbf{r}}s$ form a complete basis, wave function $\Psi(\mathbf{r},t)$ is sufficient to represent state $\ket{\Psi(t)}$. Similarly, projecting $\Psi(t)$ in k space, we can define wave function in momentum space $\Psi(\mathbf{p},t) = \bra{\mathbf{p}}\ket{\Psi(t)}$. As postulate four states, for a particle in state $\Psi(t)$, without measurement, the only information available to us is the probability of the particle being in position $\mathbf{r}$ and momentum $\mathbf{p}$, it is not possible to know the position and momentum of the particle without measurement. Moreover, as postulate five states, immediately after the measurement of an observable, the state collapse to one of its eigenstates, meaning after a measurement, the particle's state will be changed. 