\subsection{Quantum description of particles}
Quantum mechanics, formed by six postulates, is very powerful in explaining modern experiments in atomic, molecular, optical, and condensed matter physics and so on. It also introduces counter-intuitive concepts such as photons, matter-wave, and Pauli's exclusion principle. These concepts can not be well explained without a good understanding of the postulates.  

As the first postulate states, a particle is described by its quantum state $\ket{\Psi(t)}$. $\ket{\Psi(t)}$ contains all the information about the particle and can be represented by projecting it in any complete basis. For example, $\Psi(\mathbf{r},t) = \bra{\mathbf{r}}\ket{\Psi(t)}$ is defined as the spatial wave function of the particle. Since spatial states $\{\ket{\mathbf{r}}\}$ form a complete basis, wave function $\Psi(\mathbf{r},t)$ is sufficient to represent state $\ket{\Psi(t)}$. Similarly, projecting $\Psi(t)$ in the basis of $\{\ket{\mathbf{p}}\}$, we can define wave functions in momentum space $\Psi(\mathbf{p},t) = \bra{\mathbf{p}}\ket{\Psi(t)}$. As postulate four states, for a particle in state $\ket{\Psi(t)}$, without measurements, the information associated with the classical variables $\mathbf{r}$ and $\mathbf{p}$ are the probability of the particle being in position $\mathbf{r}$ and momentum $\mathbf{p}$. It is not possible to know the position and momentum of the particle without measurements. Moreover, as postulate five states, immediately after the measurement of an observable, the state collapse to an eigenstate of the observable, meaning after a measurement, the particle's state may be changed. 