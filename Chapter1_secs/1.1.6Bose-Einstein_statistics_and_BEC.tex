\subsection{Bose-Einstein statistics and non-interacting BEC}
Quantum statistics differs from classical statistics in two aspects, indistinguishability and exchange symmetry. 

For a two-particle wave function $\psi(\mathbf{r}_1,\mathbf{r}_2)$, if the two particles are Fermions, they can not occupy the same state. $\psi(\mathbf{r}_1,\mathbf{r}_2) = -\psi(\mathbf{r}_2,\mathbf{r}_1)$, if $\psi(\mathbf{r}_1,\mathbf{r}_2) = \psi(\mathbf{r}_2,\mathbf{r}_1)$, $\psi(\mathbf{r}_1,\mathbf{r}_2) = 0$. For Bosons, this is not the case. Any number of Bosons can occupy the same state and the system has the lowest energy when all the Bosons are in the ground state.

The Bose-Einstein distribution can be derived from the principle of maximum entropy. A Boson system with $N$ particles are distributed to states $\{\ket{\epsilon_i}\}$, each state $\ket{\epsilon_i}$ has degeneracy $g_i$ and it is occupied by $n_i$ particles. The number of micro-state is
\begin{equation}
    \Omega(\{n_i\}) = \prod_{i} \frac{(n_i+g_i-1)!}{n_i!(g_i-1)!}.
\end{equation}
Maximizing $\Omega(\{n_i\})$ under the constraints $\sum_{i}n_i = N$ and $\sum_{i}n_i\epsilon_i = U$, we can derive Bose-Einstein distribution
\begin{equation}
    n_i = \frac{1}{\exp{-\alpha-\beta\epsilon_i}-1},
\end{equation}
$\alpha$ and $\beta$ are the Lagrange multipliers. From constraints $\sum_{i}n_i = N$ and $\sum_{i}n_i\epsilon_i = U$, it can be determined $\alpha$ is chemical potential $\mu$ over $k_BT$ and $\beta$ is $-1/k_BT$ where $k_B$ is Boltzmann constant and $T$ is temperature.

To find out how does this distribution lead to Bose-Einstein condensates, we need to revisit the constraints. For a system of particles in a box of volume V. The energy density of states is
\begin{equation}
    g(\epsilon) = \frac{V}{4\pi^2}\big(\frac{2m}{\hbar}\big)^{3/2}\sqrt{\epsilon}    
\end{equation}
Replacing the sum in $\sum_{i}n_i = N$ with integral, we get
\begin{equation}\label{N}
    \frac{V}{4\pi^2}\big(\frac{2m}{\hbar}\big)^{3/2}\int_0^\infty\frac{\sqrt{\epsilon}}{\exp{(\epsilon-\mu)/k_BT}-1}d\epsilon = N
\end{equation}
Denote the integral as $I(\mu)$
\begin{equation}
    I(\mu) = \int_0^\infty\frac{\sqrt{\epsilon}}{\exp{(\epsilon-\mu)/k_BT}-1}d\epsilon
\end{equation}
For Eq.~(\ref{N}) to hold, $I(\mu)$ should be a constant, and $\mu$ should be a function of $T$. As $T$ decreases, $\mu$ should increase to keep $I(\mu)$ constant. But $\mu$ can not go positive because in that case, for states $\epsilon_i-\mu < 0$, $n_i$ will be negative. So as $T$ decreases to some critical temperature $T_c$, $\mu$ increases to 0 to keep $I(\mu)$ constant. As $T$ continue decreasing, $\mu$ can not increase anymore and we have
\begin{equation}
    \frac{V}{4\pi^2}\big(\frac{2m}{\hbar}\big)^{3/2}\int_0^\infty\frac{\sqrt{\epsilon}}{\exp{(\epsilon-\mu)/k_BT}-1}d\epsilon < N
\end{equation}
Where are the missing particles? The answer is they are condensed in the ground state with zero energy. If we look at the density of state $g(\epsilon)$, it is zero for the ground state. When $T$ is high, the number of particles that are in the ground state is negligible, so the integral counts all the particles and equals $N$. When $T<T_c$, $\mu$ is zero, a significant amount of particles start to occupy the ground state, and the missed particles will cause the loss of atoms counts. 

The critical temperature can be calculated from Eq.~(\ref{N}) with $\mu=0$.
\begin{equation}
    T_c = \frac{2\pi\hbar^2}{mk_B}\Big(\frac{N}{\eta(3/2)V}\Big)^{2/3}
\end{equation}
where $\eta(x)$ is Riemann zeta function. And it can be shown the fraction of particles in the ground state is a function of $T$ and $T_c$
\begin{equation}
    \frac{N_0}{N} = 1-\Big(\frac{T}{T_c}\Big)^{3/2}.
\end{equation}




