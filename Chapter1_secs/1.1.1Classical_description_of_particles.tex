\subsection{Classical description of particles}
Classical mechanics originates from Newtonian mechanics and have different but equivalent formulations. It assumes the position and velocity of an object can be measured and kept track of at any time, the dynamics of an object or a system of objects can be sufficiently described by position and velocity. For example, in one of the equivalent formulations, Hamiltonian mechanics \cite{goldstein2002classical}, a classical system is described by a set of canonical coordinates $\vec{\mathbf{r}} = (\vec{\mathbf{p}},\vec{\mathbf{q}})$. Here $\vec{\mathbf{p}} = (p_1,p_2,\cdots,p_N)$  and $\vec{\mathbf{q}}= (q_1,q_2,\cdots,q_N)$, they are indexed by the N-dimensional frame of reference of the system. Hamiltonian $\mathcal{H} = \mathcal{H}(\vec{\mathbf{p}},\vec{\mathbf{q}},t)$ is a function of canonical coordinates and corresponds to total energy of a system. The dynamics of the system is governed by equation
\begin{equation}\label{cla ham}
\begin{aligned}
    &\frac{d\vec{\mathbf{p}}}{dt} = -\frac{\partial\mathcal{H}}{\partial\vec{\mathbf{q}}}\\
   &\frac{d\vec{\mathbf{q}}}{dt} = \frac{\partial\mathcal{H}}{\partial\vec{\mathbf{p}}}
\end{aligned}
\end{equation}
Both $\vec{\mathbf{p}}$ and $\vec{\mathbf{q}}$ evolve with equation \ref{cla ham} deterministically and for a system of classical particles, each particle's trajectory can be traced during evolution making them distinguishable from each other.