\subsection{Second quantization}
An interacting system can be described by the second quantization Hamiltonian. In the second quantization framework, particles in a system are described by creation and annihilation operators on the basis of many-body Fock states. Creation operators $\hat{a}_k^\dag$ creates a particle in the Fock state $\ket{n_1,\dots , n_k,\dots}$ while annihilation operators $\hat{a}_k$ annihilates a particle in the Fock state $\ket{n_1,\dots , n_k,\dots}$.  
\begin{equation}
\begin{aligned}
    &\hat{a}_k^\dag\ket{n_1,\dots , n_k,\dots} = \sqrt{n_k+1}\ket{n_1,\dots,n_k+1,\dots}\\
    &\hat{a}_k\ket{n_1,\dots,n_k,\dots} = \sqrt{n_k}\ket{n_1,\dots,n_k-1,\dots}
\end{aligned}
\end{equation}
With a change of basis, the spatial creation and annihilation operators can be defined as
\begin{equation}
    \begin{aligned}
    &\hat{\psi}(\Vec{r}) = \sum_k \hat{a}_k \braket{\Vec{r}}{\Vec{k}} = \sum_k \hat{a}_k \frac{\exp{i\Vec{k}\cdot\Vec{r}}}{\sqrt{V}}\\
    &\hat{\psi}^\dag(\Vec{r}) = \sum_k \hat{a}^\dag_k \braket{\Vec{r}}{\Vec{k}} = \sum_k \hat{a}_k^\dag \frac{\exp{-i\Vec{k}\cdot\Vec{r}}}{\sqrt{V}}.\\
    \end{aligned}
\end{equation}
$\hat{\psi}^\dag(\Vec{r})$ and $\hat{\psi}(\Vec{r})$ creates and annihilates a particle at spatial coordinate $\Vec{r}$. Operator $\hat{N}_k = \hat{a}_k^\dag\hat{a}_k$ counts the number of particles in the momentum state $\ket{k}$. The two-particle operator is defined as 
\begin{equation}
    \begin{aligned}
    \hat{V}_{int} &= \frac{1}{2} \sum_{i\neq j}\hat{N}_i\hat{N}_j V_{ij} + \frac{1}{2}\sum_i \hat{N}_i(\hat{N}_i-1)V_{ii}\\
    & = \frac{1}{2}\sum_{i, j}(\hat{N}_i\hat{N}_j - \hat{N}_i\delta_{ij})V_{ij}
    \end{aligned}
\end{equation}
where $V_{ij}$ is the interaction between particles in states $\ket{i}$ and $\ket{j}$. From the orthogonality of Fock states and Pauli's exclusion principle, we can derive the commutators of creation and annihilation operators for both Bosons and Fermions.
For Bosons,
\begin{equation}
    \begin{aligned}
    &[\hat{a}_i, \hat{a}_j^\dag] = \delta_{ij}\\
    &[\hat{a}_i, \hat{a}_j] = [\hat{a}_i^\dag, \hat{a}_j^\dag] = 0
    \end{aligned}
\end{equation}
where $[A,B] = AB - BA$ is the commutator of operators $A$ and $B$.
For Fermions,
\begin{equation}
    \begin{aligned}
    &\{\hat{a}_i, \hat{a}_j^\dag\} = \delta_{ij}\\
    &\{\hat{a}_i, \hat{a}_j\} = \{\hat{a}_i^\dag, \hat{a}_j^\dag\} = 0
    \end{aligned}
\end{equation}
where $\{A,B\} = AB + BA$ is the anti-commutator of operators $A$ and $B$. With the change of basis, we can also obtain the commutation and anti-commutation relations of field operators $\hat{\psi}(\Vec{r})$ and $\hat{\psi}^\dag(\Vec{r})$.
For Bosons,
\begin{equation}
    \begin{aligned}
    &[\hat{\psi}(\Vec{r}~'), \hat{\psi}^\dag(\Vec{r}~'')] = \delta^3(\Vec{r}~'-\Vec{r}~'')\\
    &[\hat{\psi}(\Vec{r}~'), \hat{\psi}(\Vec{r}~'')] = [\hat{\psi}^\dag(\Vec{r}~'), \hat{\psi}^\dag(\Vec{r}~'')] = 0.
    \end{aligned}
\end{equation}
For Fermions,
\begin{equation}
    \begin{aligned}
    &\{\hat{\psi}(\Vec{r}~'), \hat{\psi}^\dag(\Vec{r}~'')\} = \delta^3(\Vec{r}~'-\Vec{r}~'')\\
    &\{\hat{\psi}(\Vec{r}~'), \hat{\psi}(\Vec{r}~'')\} = \{\hat{\psi}^\dag(\Vec{r}~'), \hat{\psi}^\dag(\Vec{r}~'')\} = 0.
    \end{aligned}
\end{equation}
With the commutators, we can rewrite the interaction operator $\hat{V}$,
\begin{equation}
    \hat{V}_{int} = \frac{1}{2}\sum_{i,j}\hat{a}_i^\dag\hat{a}_j^\dag V_{ij} \hat{a}_j\hat{a}_i,
\end{equation}
which is valid for both Bosons and Fermions.

In general, the Hamiltonian of a system is
\begin{equation}
    \hat{H} = \sum_i \frac{\hat{\textbf{p}}_i^2}{2m} + \hat{V}(\hat{\textbf{r}}_i) + \hat{V}_{int},
\end{equation}
in the representation of field operators, it takes the form
\begin{align}
    \hat{H} = &\int d\textbf{r} \hat{\Psi}^\dag(\textbf{r})\Bigg[-\frac{\hbar^2}{2m}\grad^2 + \hat{V}(\textbf{r})\bigg]\hat{\Psi}(\textbf{r})\\
    & + \frac{1}{2}\int d\textbf{r}'d\textbf{r}''d\textbf{r}'''d\textbf{r}''''\hat{\Psi}^\dag(\textbf{r}')\hat{\Psi}^\dag(\textbf{r}'')\matrixel{\textbf{r}'\textbf{r}''}{\hat{V}_{int}}{\textbf{r}'''\textbf{r}''''}\hat{\Psi}(\textbf{r}''')\hat{\Psi}(\textbf{r}'''')
\end{align}


