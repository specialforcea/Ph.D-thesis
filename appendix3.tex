\appendix
\renewcommand{\thechapter}{C}
\chapter{Derivation of the moments of random phase factors}
\label{appendix:moments}

In Sec.~\ref{prob sec}, we derive the joint probability density of random electric fields and the probability density of intensity \ref{P of int} by calculating the moments of the random electric fields. Under our assumptions in Sec.~\ref{prob sec}, the moments of the random electric fields are proportional the moments of the random phase factors $\phi({\bf r}')$.

Under assumptions:

\begin{align}\label{gaussian C_E}
\langle \exp\left\{- i \left[\phi({\bf r}_1)-\phi({\bf r}_2)\right]\right\}\rangle &= \exp(-\frac{|{\bf r}_1-{\bf r}_2|^2}{\sigma^2}), 
\end{align} 

and

\begin{equation}
    \phi({\bf r}) \sim Uniform(0,2\pi).
\end{equation}

\begin{align}\label{gaussian C_E}
\langle \exp\left\{- i \left[\phi({\bf r}_1)-\phi({\bf r}_2)\right]\right\}\rangle &= \langle \cos\left[\phi({\bf r}_1)-\phi({\bf r}_2)\right] - i\sin\left[\phi({\bf r}_1)-\phi({\bf r}_2)\right]\rangle \\\nonumber
&=\langle \cos\left[\phi({\bf r}_1)-\phi({\bf r}_2)\right]\rangle - i\langle\sin\left[\phi({\bf r}_1)-\phi({\bf r}_2)\right]\rangle\\\nonumber
\end{align}

By symmetry,
\begin{equation}
    \langle\sin\left[\phi({\bf r}_1)-\phi({\bf r}_2)\right]\rangle = \langle\sin\left[\phi({\bf r}_2)-\phi({\bf r}_1)\right]\rangle = 0,
\end{equation}
therefore,
\begin{equation}
    \langle \cos\left[\phi({\bf r}_1)-\phi({\bf r}_2)\right]\rangle = \exp(-\frac{|{\bf r}_1-{\bf r}_2|^2}{\sigma^2}).
\end{equation}

\begin{align}
    \langle \cos\left[\phi({\bf r}_1)+\phi({\bf r}_2)\right]\rangle &=\langle \cos\left[\phi({\bf r}_1)-\phi({\bf r}_2) + 2\phi({\bf r}_2)\right]\rangle\\\nonumber
    &=  \langle \cos\left[\phi({\bf r}_1)-\phi({\bf r}_2)\right]\cos\left[2\phi({\bf r}_2)\right]\rangle\\\nonumber
    & -\langle \sin\left[\phi({\bf r}_1)-\phi({\bf r}_2)\right]\sin\left[2\phi({\bf r}_2)\right]\rangle\\\nonumber
\end{align}
$\phi({\bf r}_1)-\phi({\bf r}_2)$ is independent of $\phi({\bf r}_2)$, so
\begin{equation}
    \langle \cos\left[\phi({\bf r}_1)+\phi({\bf r}_2)\right]\rangle = 0
\end{equation}

Similarly,
\begin{equation}
    \langle \sin\left[\phi({\bf r}_1)+\phi({\bf r}_2)\right]\rangle = 0
\end{equation}

\begin{align}
    \langle \cos\left[\phi({\bf r}_1)\right]\cos\left[\phi({\bf r}_2)\right]\rangle &=\frac{1}{2}\langle \cos\left[\phi({\bf r}_1)+\phi({\bf r}_2)\right] + \cos\left[\phi({\bf r}_1)-\phi({\bf r}_2)\right]\rangle\\\nonumber
    & = \frac{1}{2}\exp(-\frac{|{\bf r}_1-{\bf r}_2|^2}{\sigma^2})\\\nonumber
\end{align}

\begin{align}
    \langle \sin\left[\phi({\bf r}_1)\right]\sin\left[\phi({\bf r}_2)\right]\rangle &=\frac{1}{2}\langle \cos\left[\phi({\bf r}_1)-\phi({\bf r}_2)\right] - \cos\left[\phi({\bf r}_1)+\phi({\bf r}_2)\right]\rangle\\\nonumber
    & = \frac{1}{2}\exp(-\frac{|{\bf r}_1-{\bf r}_2|^2}{\sigma^2})\\\nonumber
\end{align}

\begin{align}
    \langle \sin\left[\phi({\bf r}_1)\right]\cos\left[\phi({\bf r}_2)\right]\rangle &=\frac{1}{2}\langle \sin\left[\phi({\bf r}_1)+\phi({\bf r}_2)\right] - \sin\left[\phi({\bf r}_1)-\phi({\bf r}_2)\right]\rangle\\\nonumber
    & = 0\\\nonumber
\end{align}

\begin{align}
    \langle \cos\left[\phi({\bf r}_1)\right]\sin\left[\phi({\bf r}_2)\right]\rangle &=\frac{1}{2}\langle \sin\left[\phi({\bf r}_1)+\phi({\bf r}_2)\right] - \sin\left[\phi({\bf r}_2)-\phi({\bf r}_1)\right]\rangle\\\nonumber
    & = 0\\\nonumber
\end{align}