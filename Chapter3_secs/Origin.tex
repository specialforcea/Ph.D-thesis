\subsection{The Origin of SOC in a solid-state system}
Spin-orbit coupling (SOC) \cite{lin2011spin,galitski2013spin} is ubiquitous in physical systems. It alters the electronic band structure in a solid-state system and it is crucial for the spin-Hall effect \cite{kato2004observation,konig2007quantum}, topological insulators \cite{kane2005z,bernevig2006quantum,hsieh2008topological} and Majorana Fermions \cite{sau2010generic,kitaev2001unpaired}.

SOC originates from a relativistic effect. Spin is a fundamental component of electrons described by the Dirac equation. In the non-relativistic limit, the Dirac equation reduces to Schr\"odinger equation with an additional term that couples the particle's spin to its momentum. A particle with spin moves with momentum $\hbar \Vec{k} = \hbar (k_x,k_y,k_z)$ in an electric field $\Vec{E} = E_0 \hat{z}$, in its stationary frame of reference, the particle experiences a magnetic field. The Hamiltonian of the Zeeman interaction is
\begin{equation}
    \hat{H} = -\Vec{\mu} \cdot \Vec{B}_{SO} \propto \sigma_xk_y-\sigma_yk_x,
\end{equation}
where $\Vec{\mu}$ is the magnetic moment parallel to the spin $\Vec{\sigma}$ and 
\begin{equation}
    \Vec{B}_{SO} = E_0(\hbar/mc^2)(k_x,k_y,k_z).    
\end{equation}


In a condensed-matter system, there are two well-known forms of SOC depending on the symmetry of the field in the materials. The Rashba SOC \cite{bychkov1984oscillatory},
\begin{equation}
    -\Vec{\mu} \cdot \Vec{B} \propto \sigma_xk_y-\sigma_yk_x.
\end{equation}
And the Dresselhaus SOC \cite{dresselhaus1955spin},
\begin{equation}
    -\Vec{\mu} \cdot \Vec{B} \propto -\sigma_xk_y-\sigma_yk_x.
\end{equation}

The Rashba SOC originates from a lack of mirror symmetry in two-dimensional systems and the Dresselhaus SOC from a lack of inversion symmetry in bulk crystals \cite{bychkov1984oscillatory,dresselhaus1955spin,galitski2013spin}.
