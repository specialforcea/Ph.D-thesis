\subsection{Optical force}
An atomic system interacts with optical fields, which leads to stimulated emission and Rabi oscillation when the frequency of the field is near resonance. When the field is far-detuned, the atoms see a spin-independent spatial potential, the force of which drives the atoms' center-of-mass motion. Also, the atoms interact with the vacuum modes that lead to spontaneous emission. The emitted photons have nonzero momentum, so the atoms acquire the recoil momentum when the spontaneous emission happens. From the classical physics point of view, the atoms should experience a "force" in the optical fields and the definition of the "force" can be borrowed from classical physics.
\begin{equation}
    \hat{F} = \frac{d}{dt}\hat{p} = \frac{1}{ih}[\hat{p},\hat{H}]
\end{equation}
The same as classical physics, the force operator is defined as the rate of momentum change and it can be calculated for a two-level atomic system with Hamiltonian
\begin{equation}
    \hat{H} = \hbar\delta \dyad{2}{2} + \hbar(\Omega(\Vec{r}) \dyad{1}{2} + \Omega^*(\Vec{r}) \dyad{2}{1}).
\end{equation}
$\Omega(\Vec{r})$ is a function of $\Vec{r}$ for the inhomogeneous field amplitude. 
\begin{equation}
    \Vec{F} = -[\nabla,\Vec{H}(\Vec{r})] = \hbar\nabla\Omega(\Vec{r})\dyad{2}{1} + \hbar\nabla\Omega^*(\Vec{r})\dyad{1}{2}
\end{equation}
The state of the system is represented by density operator $\hat{\rho}$ and the expectation value of operator $\hat{F}$ is
\begin{align}
    \expval{\hat{F}} &= {\rm Tr}[\hat{\rho}\hat{F}]\\ \nonumber
    &=\hbar\nabla\Omega\rho_{12} + \hbar\nabla\Omega^*\rho_{21}\\\nonumber
    &=2\hbar|\Omega|^2\nabla\phi{\rm Im}\left( \frac{\rho_{21}}{\Omega}\right) + \hbar\nabla|\Omega|^2{\rm Re}\left( \frac{\rho_{21}}{\Omega}\right)\\\nonumber
\end{align}
Here, $\phi$ is the phase of the field,
\begin{equation}
    \Omega = |\Omega|e^{i\phi}.
\end{equation}
The first term is interpreted as the dissipative force and the second term the reactive force. This can be more intuitively understood when we look at the form of the force in two extreme cases.

For the steady solution of the optical Bloch equation,
\begin{align}
    &\rho_{21} = \frac{i\Omega(\rho_{22}-\rho_{11})}{\gamma_{21} - i\delta}\\\nonumber
    &\rho_{22} = \frac{R}{\gamma(\overline{n} + 1) + 2R}\\\nonumber
    &R = \frac{2\gamma_{21}|\Omega|^2}{\gamma_{21}^2 + \delta^2}\\\nonumber
\end{align}
here, $\gamma_{21}$ is the decay rate of coherence and $R$ is the optical pumping rate, the rate the atoms are pumped from the ground state to the excited state. When the field is weak,
\begin{equation}
    |\Omega| \ll \gamma
\end{equation}
the time atoms spend in the excited state is close to zero, $\rho_{22} \approx 0$. The coherence $\rho_{21}$ can be approximated with
\begin{equation}
    \rho_{21} = \Omega\frac{\delta - i\gamma_{21}}{\gamma_{21}^2 + \delta^2}
\end{equation}
The dissipative force takes the form
\begin{align}
    F_{dis} &= 2\hbar|\Omega|^2 \Vec{k}\frac{\gamma_{21}}{\gamma_{21}^2 + \delta^2}\\\nonumber
    &= \hbar\Vec{k}R\\
\end{align}
Intuitively, it means every time the atom is pumped from the ground state to the excited state, it acquires momentum $\hbar\Vec{k}$. When spontaneous emission happens, the photon is emitted in any direction with equal probability, on the average, the atom does not acquire momentum in spontaneous emission. 

In the other case, when $\delta$ is large, 
\begin{equation}
    \delta \gg |\Omega|, \delta \gg \gamma_{21}    
\end{equation}
Optical pumping rate $R \approx 0$ and the population in the excited state $\rho_{22} \approx 0$. 
\begin{equation}
    {\rm Im}\left[\frac{\rho_{21}}{\Omega}\right] \approx \frac{1}{\delta}
\end{equation}
and the reactive force takes the form
\begin{equation}
    F_{react} = \frac{\hbar\nabla|\Omega|^2}{\delta}.
\end{equation}
Effectively, the atoms see a potential
\begin{equation}
    V(\Vec{r}) = -\frac{\hbar|\Omega(\Vec{r})|^2}{\delta}
\end{equation}
The potential is spin-independent, its strength is proportional to the intensity of the field $|\Omega\Vec{r}|^2$ and the sign of $\delta$ determines if the potential is repulsive or attractive. When $\delta > 0$, the field is red detuned, and the potential is lower for stronger intensity, so the potential is attractive. When the field is blue detuned, it is repulsive.  

This potential is known as the dipole potential, it originates from the AC stark shift. Due to the large detuning, it does not drive the transition of internal states but changes the energy of the ground state so it is effectively a spin-independent potential. Dipole potential is widely used in atomic physics experiments. It can be used as an attractive trap, called a dipole trap, to trap atoms, to do evaporative cooling, and to make an optical lattice. 
The repulsive potential is also very useful, it can be used to make a 1D trap when the laser is in Laguerre-Gaussian mode \cite{salces2018equations} and make random repulsive potential (optical speckle) which is discussed in detail in Ch.~(\ref{speckle_chapter}).

    

