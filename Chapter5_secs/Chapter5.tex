\renewcommand{\thechapter}{5}

\chapter{Enhanced transport of SOC Bose gases in disordered potentials: model and simulation}
In materials, microscopic electron scattering processes partly govern the macroscopic conductivity and AL predicts a metal-insulator transition. Increasing a system's conductivity therefore requires some change in these scattering processes.  The most straightforward mechanism is to reduce the disorder strength. Here we describe an alternate approach in which spin-orbit coupling (SOC) greatly suppresses the back scattering and thereby increases the conductivity.  We then propose a realization of this effect using a cold-atom Bose-Einstein condensate (BEC) with laser-induced SOC~\cite{lin2011spin} and disorder from optical speckle.

SOC is a ubiquitous phenomenon in physical systems that describes the interaction between a particle's spin and its momentum. When SOC is combined with a transverse magnetic field (in the sense of Zeeman shifts, not Lorentz forces), gaps in the dispersion relation can open at spin-degeneracy points.  The opening of these gaps modifies the electrons' scattering processes and affects transport. AL was first realized for ultracold-cold atomic systems~\cite{billy2008direct,roati2008anderson}  in 2008, and the experimental techniques are now well established.  Shortly thereafter,  techniques for creating SOC in the cold atom lab were demonstrated~\cite{lin2011spin}.  Together, this makes cold-atom systems an ideal platform to study the interplay between AL and SOC. 

Optical speckle is a powerful tool for creating disordered potentials for atomic systems~\cite{clement2006experimental}.  The strength of the resultant potential is under direct experimental control: the spatial correlation length is tunable and the correlation function is well known.  Here we analytically and numerically study backscattering in speckle potentials of quasi-1d spin-orbit coupled (SOBECs) and compare to the case without SOC. We show that SOC can reduce the scattering processes for specific momentum states. In the broader context, our results suggest that in thin nano-wires, SOC might significantly decrease resistance and improve energy efficiency in electronic devices.

In this chapter, we analytically calculate the probability an initial momentum state being scattered by the speckle potential to any final momentum state and show that SOC can reduce back scattering. Next, we describe numerical simulations of quasi-1d BECs starting in different momentum states subject to a speckle potential with and without SOC. We show that even with the higher order scattering processes and interaction between particles present in the numerical simulations, SOC can reduce the localization effects of disorder and enhance transport.

\section{Scattering of an SOBEC from a speckle potential} \label{Model}

\begin{figure}[htbp]
    \centering
    \includegraphics{Chapter5_secs/Fig3_Dispersions.pdf}
    \caption{Fermi's Golden Rule. Momentum are expressed in units of the single-photon recoil momentum $\kr$ used to create SOC in (c). (a) Representative PSD for optical speckle with $k_c = 6 \kr$. (b) Free particle dispersion relation. The dashed arrow marks the boundary above which the FGR rate vanishes, while the solid arrow provide an example with non-zero rate. (c) SOC dispersion relations computed for $\delta = 0$ add $\Or = 1\Er$ colored according to the expectation value $\langle\sigma_z(q)\rangle$, with arrows marked as in (b).  Note the transition through the gap in the dispersion relation at $E\approx \Er$ where the FGR rate is nearly zero.}
    \label{fig:Dispersion relations}
\end{figure}

We now focus on the motion of spin-orbit coupled bosons in a speckle-induced disorder potential.  In this section, we develop a Fermi's golden rule (FGR) approach for scattering from a disorder potential, both with and without SOC, schematically depicted in Fig.~\ref{fig:Dispersion relations}.   The first order scattering processes captured by the FGR are possible when a matrix element (here from the disorder potential) can couple energetically degenerate initial and final states (here momentum or quasi-momentum states).  We will see that the strength of this coupling is proportional to the PSD of the speckle potential, an example of which is shown in Fig.~\ref{fig:Dispersion relations}(a).  As depicted in  Fig.~\ref{fig:Dispersion relations}(b), this implies an absence of scattering for momenta differences larger than the speckle-cutoff $k_c$.  Adding SOC, as in Fig.~\ref{fig:Dispersion relations}(c), can suppress scattering for additional wavevectors.  Because a spin-independent speckle potential has no spin-changing matrix element, the energetically allowed transition at an energy $E/\Er \approx 1$ between states of opposite spin is strongly suppressed.  The following discussion quantifies these observations.

\subsection{Spinless atoms}\label{sec_spinless_atoms}

For spinless free particles, the unperturbed Hamiltonian $H = \hbar^2 k^2 / 2m$ implies that we study scattering between initial and final momentum states, labeled by  $\ket{k_0}$ and $\ket{k_f}$ respectively.  Figure~\ref{fig:Dispersion relations}(b) depicts examples by open circles, with arrows connecting initial states to final states.

The time evolution of the initial state $\ket{\psi(0)} = \ket{k_0}$ subject to the speckle potential $V(x)$ may always be expressed as
\begin{align} 
\ket{\psi(t)} &= \sum_{k}C_{k,k_0} (t) e^{-i\omega_k t}\ket{k}, \label{expand}
\end{align}
with $C_{k,k_0}(0) = \delta_{k,k_0}$ and $\hbar\omega_k = \hbar^2 k^2 / 2 m$.  The coefficients $C_{k,k_0} (t)$ are governed by the time-dependent Schr\"{o}dinger equation giving the exact expression
\begin{align}
    C_{k_f,k_0}(t) = &  C_{k_f,k_0}(0) + \label{allorder} \\
    & \frac{1}{i\hbar}\sum_k \bra{k_f}\hat V\ket{k} \int_0^t d\tau e^{i\omega_{k_f,k}\tau}C_{k,k_0}(\tau).\nonumber
\end{align}
with 
\begin{align}
    \omega_{k,l} &= \omega_k-\omega_l, &{\rm and} && \hat V &= \sum_x V(x) \ket{x}\bra{x}.
\end{align}
An order-by-order perturbation theory is typically obtained by recursively inserting the integral expression for $C_{k_f,k_0}(t)$ back into the integrand; unfortunately, the general problem is intractable and we truncate the perturbation series at first order. This term is effectivly obtained by replacing $C_{k_f,k_0}(\tau)$ with $C_{k_f,k_0}(0) = \delta_{k_f,k_0}$, and find
\begin{equation}\label{firstC}
    C_{k_f,k_0} (t) = \delta_{k_f,k_0} + \frac{1}{i\hbar}\int_0^t d\tau\bra{k_f}\hat V\ket{k_0} e^{i\omega_{k_f,k_0}\tau}.
\end{equation}
Unfortunately, we do not know $V(x)$ for any specific realization of the speckle potential.

In Chpt.~\ref{speckle_chapter} we characterized optical speckle in terms of second-order statistical metrics such as the PSD, here equal to $\rho(k_f-k_0) = \langle\bra{k_f}\hat V\ket{k_0}\bra{k_0}\hat V\ket{k_f}\rangle$, where the double-brackets indicate the ensemble average. The resulting ensemble averaged transition probability
\begin{equation}
 P_{f,0} (t) = \frac{\rho(k_f- k_0)}{\hbar^2}\left[\frac{2}{\omega_{f,0}} \sin\left(\frac{\omega_{f,0}t}{2}\right)\right]^2
\end{equation}
is a sharply peaked function centered at $\omega_{f,0} = 0$ with width $2\pi/t$, showing that a narrow range of energy matching states can be populated.  For long times, $\omega_{f,0}t \gg 1$ the quantity in square brackets converges to a scaled Dirac delta function $t\times\delta(\omega_{f,0})$.

Figure~\ref{fig:Dispersion relations}(a) displays the normalized PSD for a speckle potential computed with $k_c = 6 \kr$, reminding us that $\rho(k) = 0$ for $k\geq k_c$.  Our FGR expression allows two types of scattering processes for the free particle dispersion shown in Fig.~\ref{fig:Dispersion relations}(b).  In the first process, depicted by the black arrow, the atom's initial momentum is reversed, changed by $\Delta k = 2 k_0$; as indicated by the dashed line, this process is second-order forbidden for $k_0 \geq k_c/2$.  In the second process (not pictured), the atom's momentum is only infinitesimally changed: spreading the wave-packet, but leaving the average momentum unchanged.
This picture shows that backscattering is essential for momentum-relaxation.
\subsection{Spin-orbit coupled atoms}

Our 1D SOC coupling~\cite{lin2011spin} is created by illuminating a two-level atom with a pair of counter-propagating lasers with wavelength $\lambdar$ tuned to drive stimulated Raman transitions between states $\{ \ket{q+\kr,\uparrow},\ket{q-\kr,\downarrow} \}$.  Here $\hbar\kr = 2\pi\hbar/\lambdar$ and $\Er = \hbar^2 \kr^2 / 2m$ are the single-photon Raman recoil momentum and energy respectively.
Subject to this Raman coupling, the atoms obey the 1D Hamiltonian \begin{align}
\hat{H}(q) = \left[\frac{\hbar^2 q^2}{2m} + \frac{\hbar^2 \kr^2}{2m}\right]\hat 1 + \left(\frac{\hbar^2 \kr q}{m} + \frac{\delta}{2}\right)\hat \sigma_z + \frac{\hbar\Or}{2}\hat \sigma_x,\label{eq:soc}
\end{align}
where $\left\{\hat 1, \hat\sigma_x,\hat\sigma_y,\hat\sigma_z\right\}$ are the identity and Pauli operators, respectively.
Here $q$ is the quasi-momentum, $\Or$ is Raman coupling strength, and $\delta$ is the detuning from the two-photon Raman resonance condition.
The resulting dispersion relations, plotted in Fig.~\ref{fig:Dispersion relations} for $\delta = 0$ and $\Or = \Er$, have energies $E^\pm(q)$  labeled by $q$ along with $\pm$ to indicate if they are in the upper or lower band.

These new energies and their associated amplitudes
\begin{align*}
\ket{q,\pm} &\propto a_\pm(q)\ket{q-\kr,\downarrow} + b_\pm(q)\ket{q+\kr,\uparrow}
\end{align*}
change the potential scattering processes, which we again compute using a FGR expression.  The coefficients
\begin{align*}
a_\pm(q) & = \pm\frac{\Or}{2}& {\rm and} && b_\pm(q) &= \pm\frac{\Delta(q)}{2} + \frac{\sqrt{\Delta^2(q) + \Or^2}}{2}.
\end{align*}
along with the quasi momentum dependent detuning
\begin{align}\label{delta}
    \Delta(q) &= \frac{2\hbar^2q\kr}{m} + \delta
\end{align}
fully define these superposition states.

Following the same FGR argument presented above for initial states $\ket{q_0,-}$ in the lower dispersion scattering from a spin-independent speckle potential, we find scattering probabilities
\begin{align}\label{transfer prob}
    P^\pm_{f, 0}(t) &= \frac{\rho(\Delta q)}{\hbar^2}\left|\frac{2\sin(\omega^\pm_{f, 0}t)}{\omega^\pm_{f, 0}}\!\matrixel{q_f,\pm}{e^{i\Delta q x}}{q_0,-}\right|^2
\end{align}
expressed in terms of the quasimomentum and energy differences $\hbar\Delta q = \hbar q_f-\hbar q_0$ and $\hbar \omega_{f,0}^\pm = E^\pm(q_f) - E^-(q_0)$.  For most initial states $\ket{q_0,-}$, such as two higher higher-energy states marked in Fig.~\ref{fig:Dispersion relations}(c), the scattering is essentially unchanged from our spinless example, with scattering occurring between energy-matched states with the same initial and final spin.  In contrast, for initial states residing in the SOC energy gap there is no energy-matched state of the same spin available for back-scattering; as indicated by the dashed line scattering is greatly suppressed.  We note that that backscattering is not completely blocked, because the energy matching states $\ket{\pm q_0,-}$ are not spin-eigenstates and do have some spin-overlap.
\subsection{Computed scattering rates}

\begin{figure}[tbp]
    \centering
    \includegraphics{Chapter5_secs/Fig4_Scattering_Rate.pdf}
    \caption{First order average scattering Rate. The right panel is the normalized average scattering rate for the initial 13.4ms, for each initial state $\ket{q_0,\downarrow'}$ on the right side of the bottom band. The left panel shows the corresponding dispersion relation, for coupling strength $\Or = 0, 0.5 \Er, 2.5\Er$ and $ 4.0\Er$. } %For small $\Or$, the range of $\ket{q_0,\downarrow'}$s in GPE simulation covers two peaks where first order scattering is strong, the top of SOC lower band and bottom of SOC upper band. Between the two peaks is the SOC gap where first order scattering is nearly zero. At $\Or=2.5\Er$, the left peak starts to fall out of the simulation range and when $\Or \geq 4.0\Er$, the left peak disappears. Right panel are the corresponding SOC band structures. }
    \label{fig:scatteringRate}
\end{figure}

We now use these FGR expressions to compute the scattering rates for both forward scattering and back scattering processes.  Because we are interested in transport properties, we define forward scattering processes as those that leave the sign of the group velocity unchanged and back scattering processes and those that do reverse the direction of motion.
We therefore consider initial states $\ket{q_0,-}$ in the lower band with positive group velocity.  Because the lower energy SOC dispersion plotted in Fig.~\ref{fig:scatteringRate} can have a pair of minima located at $\pm q_{\rm min}$, we always select $q_0 > q_{\rm min}$ to assure positive group velocity.  We numerically evaluated the FGR for $^{87}{\rm Rb}$ atoms illuminated with $\lambdar=790\ {\rm nm}$ Raman lasers, giving $\Er = h \times 3.7\ {\rm kHz}$, and for speckle with $k_c = 6 \kr$.  The $t=13.4\ {\rm ms}$ interaction time was selected to be experimentally relevant.  

The right panels of Fig.~\ref{fig:scatteringRate} show the normalized scattering rate computed for four different values of $\Or$, with the back-scattering rate plotted in back and forward scattering plotted in gray.  These rates combine the contributions from the $\pm$ bands in Eq.~(\ref{transfer prob}).

Panel (a), computed for $\Or=0$ (equivalent to the case with no SOC), shows two key effects.  Firstly, the diverging forward and back scattering rates at low energy follow from the diverging density of states (DoS) in 1D.  Secondly, as expected, the rate of back scattering falls to zero when $\delta q > k_c$.

Panels (b) and (c) show cases with a well resolved SOC energy gap.  As we hoped for, back scattering is nearly completely suppressed for initial energies in the energy gap, while forward scattering is hardly changed.  In addition a pair of singular features boarder the energy gap, resulting from the diverging DoS the the local extrema of the dispersions.  Panel (d) shows the same phenomena, but just as the two minima at $\pm q_{\rm min}$ have merged into a single minimum at $q_{\rm min}=0$.

We therefore conclude, for non-interacting particles back scattering and momentum relaxation is nearly completely suppressed for atoms starting in the SOC energy gap.
\section{Numerical simulation of GPE}\label{GPE_simulation}

Our single particle FGR results only describe short-time scattering from a disorder potential, they cannot describe the full approach to equilibrium.
To bridge the gap between the FGR and the physical system, we need to account for both higher order scattering processes and interparticle interactions.  In our proposed SOBEC realization all aspects of SOC Hamiltonian and the speckle potential are tunable, making SOBECs an ideal system for exploring enhanced transport in 1D quantum wires.

\subsection{Gross-Pitaevskii equations}
Here we numerically study the deceleration of an SOBEC initially moving in a speckle potential using the time-dependent Gross-Pitaevskii equation (GPE).
The time-dependent GPE 
\begin{align}\label{gpe}
i\hbar\partial_t\Psi({\bf r},t) &= \left[-\frac{\hbar^2}{2m}\nabla^2 + V({\bf r}) + g_{\rm 3D}|\Psi({\bf r},t)|^2\right]\Psi({\bf r},t)
\end{align}
is a non-perturbative dynamical description ~\cite{erdHos2010derivation} of a large number of interacting identical bosons occupying the same spatial mode $\Psi({\bf r}, t)$, normalized to the total atom number, $N=\int d^4{\bf r} |\Psi({\bf r}, t)|^2$
The interaction strength $g_{\rm 3D} = 4\pi\hbar^2a_s/m$ can be expressed in terms of the $s$-wave scattering length $a_s$.
This GPE provides a good description of low-temperature spin-polarized BECs, with negligible thermal excitations~\cite{Dalfovo1999}. 

Since our focus is on 1D transport, we must first obtain a 1D description of our 3D system~\cite{bao2003numerical}.
Here we, we assume that the potential $V({\bf r}) = V(x) + V_\perp(y,z)$ can be separated into a weak longitudinal potential $V_\parallel(x)$ along with a strongly confining transverse potential $V_\perp(y,z)$.
When the single-particle energy spacing from $V_\perp(y,z)$ greatly exceeds all other energy scales, the 3D wavefunction can be factorized into
\begin{equation}
    \Psi({\bf r},t) = \psi(x,t)\phi(y, z),
\end{equation}
containing a longitudinal term of interest giving the 1D density $n(x) = |\psi(x,t)|^2$, and a transverse term $\phi(y, z)$, normalized to unity, assumed to be the ground state of the transverse potential. 
Inserting this ansatz into the GPE and integrating out the transverse degrees of freedom, gives the 1D GPE
\begin{equation}
    i\hbar\partial_t\psi = \left[-\frac{\hbar^2}{2m}\partial_x^2 + V(x) + g|\psi|^2\right]\psi,
\end{equation}
suitable for studying single-component 1D bosons in a speckle potential with 1D interaction strength 
\begin{equation}
    g = g_{\rm 3D}\int dy dz |\phi(y,z)|^4.
\end{equation}
For compactness of notation, here and below, we shall omit the functional dependance of $\psi$ on $x$ and $t$.


The two component 1D spinor GPE describing SOBECs extends Eq.~\eqref{eq:soc} to include interactions, and consists of a pair of coupled non-linear differential equations
\begin{align}
i\hbar\partial_t\psi_\uparrow &= \left[\frac{\hbar^2}{2m}\left(-i\partial_x + \kr\right)^2 + \frac{\delta}{2} + V(x) + g_{\uparrow\uparrow} |\psi_\uparrow|^2 + g_{\uparrow\downarrow}|\psi_\downarrow|^2\right]\psi_\uparrow + \frac{\Or}{2}\psi_\downarrow \\
i\hbar\partial_t\psi_\downarrow &= \left[\frac{\hbar^2}{2m}\left(-i\partial_x - \kr\right)^2 - \frac{\delta}{2} + V(x) + g_{\downarrow\downarrow} |\psi_\downarrow|^2 + g_{\uparrow\downarrow}|\psi_\uparrow|^2\right]\psi_\downarrow + \frac{\Or}{2}\psi_\uparrow
\end{align}
including the interaction strengths $g_{\uparrow\uparrow}$, $g_{\uparrow\downarrow}$, and $g_{\downarrow\downarrow}$.
Here we focus on the specific case of $^{87}{\rm Rb}$ atoms~\cite{kawaguchi2012spinor} in the $f=1$ ground state manifold and have selected $\ket{\uparrow} = \ket{m_F=0}$ and $\ket{\downarrow} = \ket{m_F = -1}$. 
The interactions can be parameterized in terms of an $s$-wave pseudo-potential $(g_{0, {\rm 3D}} + g_{2, {\rm 3D}}\Vec{F}_\alpha\cdot \Vec{F}_\beta) \delta({\bf r}_i - {\bf r}_j)$ now dependent on spin.
In $^{87}{\rm Rb}$'s $f = 1$ manifold $g_{0, {\rm 3D}} = 100.86 \times 4\pi\hbar^2 a_{\rm B}/m$ is vastly larger than $g_{2, {\rm 3D}}\approx-4.7\times10^{-3} \times g_{0, {\rm 3D}}$, where $a_{\rm B}$ is the Bohr radius~\cite{vanKempen2002,Widera2006}.
The interaction coefficients reduce to effective 1D interaction strengths just as in the single component case, and are related to the generic coefficients~\cite{Ho1998,Ohmi1998} via $g_{\uparrow\uparrow} = g_0$ and $g_{\downarrow\downarrow}  = g_{\uparrow\downarrow} = g_0 + g_2$.
Table \ref{table:1} summarizes the parameters used in our simulations. 


Our simulation results are divided into two sections: Sec.~\ref{single} hones our understanding by considering a single-component BEC evolving in a speckle potential, and then in Sec.~\ref{soc} we contrast to the case with SOC. 
In both sections, we simulate initially trapped BECs accelerated to an initial momentum $k_0$ or quasi-momentum $q_0$ and we study their deceleration. 
All the results are averaged over 20 speckle realizations, as in  Fig.~\ref{fig:Optical speckle}(b). 
The average speckle potential $h\times 200\ {\rm Hz} \approx 0.05\Er$ was selected to be weak enough to cause no trapping effect yet strong enough to produce significant deceleration within $15 {\rm ms}$.

\begin{table}
\renewcommand{\arraystretch}{1.25}
\begin{tabular}{l c c}
\hline\hline
Description & Symbol & Value \\
\hline
%interaction coefficient & $g_{0, {\rm 3D}}$ & $8.7\times10^{-52}$~${\rm J} \cdot {\rm m}$\\
%interaction coefficient & $g_{2, {\rm 3D}}$ & $-4.0\times10^{-54}$~${\rm J} \cdot {\rm m}$\\
$^{87}{\rm Rb}$ atomic mass & $m$ & $1.42\times10^{-25}~{\rm kg}$\\
Raman laser wavelength & $\lambdar$ & $790~{\rm nm}$ \\
Recoil energy & $\Er$ & $h\times 3.678\ {\rm kHz}$\\
Dipole trap frequency & $\omega/2\pi$ & $10~{\rm Hz}$  \\
Angle of Raman beams & $\theta_{\rm R}$& $180^{\circ}$ \\
Speckle potential cut off &  $k_c$ & $6\kr$\\
Average speckle potential & $\overline{V(x)}$ & $0.05\Er$\\
Grid spacing  & $\delta x$ & $66\ {\rm nm}$\\
Grids points (single-component)& $N_x$ & $2^{14} + 1$\\
Grids points (SOC)& $N_x$ & $2^{13} + 1$\\
Atom number & $N$ & $2\times 10^5$ \\
Chemical potential & $\mu$ & $h\times 300\ {\rm Hz}$ \\
% 300Hz, TF radius 34um.
\hline\hline
\end{tabular}
\normalsize
\caption{Simulation parameters}
\label{table:1}
\end{table}


\subsection{Single component systems}\label{single}

\begin{figure}[tbp]
    \centering
    \includegraphics{Chapter5_secs/Fig5_Single_Localization.pdf}
    \caption{Single-component GPE simulation with $k_c / \kr = 6$.  
    (a). Representative disorder potential used in our simulations. The inset shows an expanded view with visible structure.
    (b) Density distributions.  The filled red curve depicts the initial density distribution, while the black and red curves show the final-state density distributions for initial momenta $\ket{k_0=0.2\kr}$ and $\ket{k_0=3.1\kr}$,above and below $k_c/2$, respectively.
    (c) Mean momentum. $\langle k(t) \rangle$ averaged over 20 speckle realizations is plotted for a range of initial momentum in the range of $0$ to $3.3\kr$, the $t=0$ point of each curve marks the initial $k_0$. 
    (d) Deceleration.  The colored symbols plot $k_f = \langle k(t=16\ {\rm ms}) \rangle$ as a function of $k_0$ along with their standard deviations, and the black line marks $k_f = k_0$ corresponding to ballistic motion.
    }
    \label{fig:single}
\end{figure}

The simulations are performed in three steps to as accurately as possible model a realistic experimental sequence. 
First, we initialize a ground state BEC in a harmonic trap using imaginary time evolution~\cite{chiofalo2000ground}, giving the density distribution plotted as the filled red curve in Fig.~\ref{fig:single}(b)], and follow with real-time evolution.
Second, because the BEC's narrow momentum distribution is centered at $k=0$, we briefly apply a linear potential $\alpha x$ with time-evolution approximately described by the phase factor $\exp(i k_0 x)$, a momentum translation operator that transforms $\ket{k=0}$ to $\ket{k_0}$. 
Third, having prepared our $\ket{k_0}$initial state, we replace the harmonic potential with a speckle potential (with $k_c/\kr = 6$) and follow the time evolution for $16\ {\rm ms}$. Fig.~\ref{fig:single}(a) shows a representative disorder potential that we use in the simulations.
Fig.~\ref{fig:single}(b) captures the main result of this section: when $k_0 > k_c / 2$ the time-evolution is almost unchanged by the speckle potential, while slowly moving initial states are both decelerated and exhibit considerable interference.

Figure~\ref{fig:single}(c) plots the ensemble-averaged momentum $\langle k(t)\rangle$ as a function time for a range of initial states with $k_0$ from near-zero to $k_0 / \kr = 3.3$, and Fig.~\ref{fig:single}(d) plots the final momentum $k_f$ as a function of $k_0$.
At $t=0$, the average momentum is $\langle k(t)\rangle = k_0$; for $k_0\gtrsim k_c/2$ the BEC evolves ballistically, leaving $\langle k(t)\rangle$ unchanged, while $\langle k(t)\rangle = k_0$ falls rapidly for smaller $k_0$.
Both of these observations are consistent with our FGR analysis which showed a complete absence of momentum-changing backscattering for $k_0 \geq k_c/2$, and with rapidly increasing backscattering as $k$ falls to zero.
\subsection{SOBECs}\label{soc}

\begin{figure}[htbp]
    \centering
    \includegraphics{Chapter5_secs/Fig3_make_eigen.pdf}
    \caption{Make SOC eigenstates $\ket{q_0, -}$. (a). The SOC dispersion relation with $\delta = 2\hbar^2 (q_0 - q_{\rm min}) \kr/m$, $\Or = \Er$. The dashed line indicates the detuning $\Delta(q_m)$ between $\ket{q_m, -}$ and $\ket{q_m, +}$. (b). The SOC dispersion relation with $\delta = 0$, $\Or = \Er$. The dashed line indicates the detuning $\Delta(q_0)$ between $\ket{q_0, -}$ and $\ket{q_0, +}$, with $q_0=2.0 \kr$ as an example. $\delta$ in (a) is tuned such that $\Delta(q_m)$ is equal to $\Delta(q_0=2.0\kr)$ in (b). }
    \label{fig:SOC_eigen}
\end{figure}

As in the single-component case, simulations with SOC begin with three steps aligned with experiment, however, the process of preparing the initial quasimomentum state $\ket{q_0,-}$ is considerably more elaborate than preparing a momentum state $\ket{k_0}$ in a single component system.
(1) As before, we initialize a ground state BEC in a harmonic trap using imaginary time evolution, spin polarized in state $\ket{k_0 = 0, \downarrow}$.
(2) We then use a combination of adiabatic and unitary evolution (described below) to transform this state into $\ket{q_0,-}$ for $\delta=0$ and $\Or$ ranging from $0.5\Er$ to $8\Er$.
(3) Lastly, we again remove the harmonic potential and again follow the time evolution with a speckle potential ($k_c/\kr = 6$) for $16\ {\rm ms}$.

Our procedure (2) begins with the observation that in a frame moving with velocity $\hbar \delta k / m$, the detuning $\delta$ present SOC Hamiltonian in Eq.~\eqref{eq:soc} is Dopper-shifted~\cite{Cheuk2012, Valdes-Curiel2017} to $\delta + 2\hbar^2 \delta k \kr/m$.
Our first task is to adiabatically transform the initial state $\ket{k_0 = 0\downarrow}$ into $\ket{q_{\rm min}, -}$, a ground state SOBEC with quasi-momentum centered at $q=q_{\rm min}$. $\ket{q_{\rm min}, -}$ is the global minima of the SOC dispersion, but with $\delta = 2\hbar^2 (q_0 - q_{\rm min}) \kr/m$. Fig.~\ref{fig:SOC_eigen}(a) shows the SOC dispersion for $\Or = \Er$, $\delta = 2\hbar^2 (q_0 - q_{\rm min}) \kr/m$ and $q_0 = 2.0 \kr$. Fig.~\ref{fig:SOC_eigen}(b) shows the SOC dispersion for $\Or = \Er$, $\delta = 0$. In both Fig.~\ref{fig:SOC_eigen}(a) and Fig.~\ref{fig:SOC_eigen}(b), the dashed lines indicate the detuning $\Delta(q)$. $\delta$ in Fig.~\ref{fig:SOC_eigen}(a) is determined such that $\Delta(q_m)$ in Fig.~\ref{fig:SOC_eigen}(a) and $\Delta(q_0 = 2.0 \kr)$ in Fig.~\ref{fig:SOC_eigen}(b) are equal.
We achieve this by ramping up the Raman coupling strength from zero to $\Or$ on a time scale slow compared to $\hbar/\Delta(q_0)$.
In the slow ramp up process, the harmonic trap provides the restoring force required to keep the state at a local minima of the dispersion~\cite{lin2011spin}, i.e., with zero group velocity. 
Lastly, we diabatically set $\delta=0$ and apply momentum kick $\exp[i(q_0-q_m)x]$, giving the desired state $\ket{q_0, -}$. 

\begin{figure}[htbp]
    \centering
    \includegraphics{Chapter5_secs/Fig6_SOC_Localization.pdf}
    \caption{Motion in the presence of speckle and SOC.
    The left column was computed without interactions and the right column added interactions. 
    (a) and (b) Density distributions colored by their magnetization according to the color bar in Fig.~\ref{fig:Dispersion relations}.
    The shaded curve depicts the initial density distribution, while the remaining red and and black curves were computed for  $q_0=2.0k_{\rm R}$ (in the SOC gap) and $q_0=1.2k_{\rm R}$ (below the SOC gap), respectively.
    (c) and (d) Ensemble averaged final group velocity plotted as a function of initial group velocity for coupling strengths  from $0.5E_{\rm R}$ to $7.5E_{\rm R}$, spaced by $1.0E_{\rm R}$.
    The results in (c) and (d) were averaged over 20 random speckle realizations. 
       }
    \label{fig:SOC}
\end{figure}

Figures~\ref{fig:SOC}(a) and (b) show representative density distributions $n(x) = |\psi_\uparrow(x,t)|^2 +|\psi_\downarrow(x,t)|^2$ before and after a $16\ {\rm ms}$ time evolution with $\Or = 2\Er$, both (a) with no interactions and (b) with interactions.
In both cases the pink shaded curve depicts the initial density distribution, while the density distributions for initial quasimomemta of $q_0 = 1.2 \kr$ and $2.0\kr$ are shown by the black and red curves respectively.
In both cases the momentum exchange for back-scattering is below $k_c$, however, as with the FGR results, this direct simulations shows that the initial state prepared with energy within the SOC energy gap experiences negligible change in velocity, independent of the presence of interactions.

While the free particle group velocity is simply related to wave-vector by $v = \hbar k / m$, atoms evolving according to the SOC dispersions, as in Fig.~\ref{fig:scatteringRate}, have group velocity given by the more complex relation
\begin{align}
\frac{v_\pm}{\vr} &= \frac{q}{\kr} \left\{1 \pm \left[\left(\frac{q}{\kr}\right)^2 + \left(\frac{\Or}{4 \Er},\right)^2\right]^{-1/2}\right\},
\end{align}
for atoms in state $\ket{q, \pm}$, expressed in units of the recoil velocity $\vr = \hbar \kr/m$.
Because we are interested in transport phenomena, it is this group velocity not the quasimomentum, that is the quantity of primary interest.

Fig.~\ref{fig:SOC}(c) and Fig.~\ref{fig:SOC}(d) plot the final group velocity $v_f$ as a function of initial group velocity $v_0$ after a $16\ {\rm ms}$ period of free evolution, both (c) with no interactions and (d) with interactions, and with $\Or$ from $0.5\Er$ to $7.5\Er$.
As compared to the simulations without SOC in Fig.~\ref{fig:single}(d), these curves show a near-complete suppression of relaxation for velocities near $v_0 \approx \vr$, in the SOC energy gap, and with an increasing window of suppression with increasing $\Or$.

Lastly, we see that interaction effects do play a role, leading to more rapid deceleration.
The origin of this effect can be understood by comparing the red curves in Figs.~\ref{fig:SOC}(a) and \ref{fig:SOC}(b): adding interactions leads a mean-field driven expansion of the BEC, increasing the range of velocities present.
As a result, when the SOC energy gap is small (small $\Or$), a significant fraction of the BEC's velocity distribution falls outside the SOC energy gap, thereby sampling points in the dispersion where first-order backscattering is allowed.
At larger $\Or$, motion is near-ballistic near the center of the SOC gap, but the transition from ballistic to decelerated is smoothed as compared to the case with no interactions.
