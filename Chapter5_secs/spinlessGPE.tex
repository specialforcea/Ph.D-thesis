\subsection{Single component systems}\label{single}

\begin{figure}[tbp]
    \centering
    \includegraphics{Chapter5_secs/Fig5_Single_Localization.pdf}
    \caption{Single-component GPE simulation with $k_c / \kr = 6$.  
    (a). Representative disorder potential used in our simulations. The inset shows an expanded view with visible structure.
    (b) Density distributions.  The filled red curve depicts the initial density distribution, while the black and red curves show the final-state density distributions for initial momenta $\ket{k_0=0.2\kr}$ and $\ket{k_0=3.1\kr}$,above and below $k_c/2$, respectively.
    (c) Mean momentum. $\langle k(t) \rangle$ averaged over 20 speckle realizations is plotted for a range of initial momentum in the range of $0$ to $3.3\kr$, the $t=0$ point of each curve marks the initial $k_0$. 
    (d) Deceleration.  The colored symbols plot $k_f = \langle k(t=16\ {\rm ms}) \rangle$ as a function of $k_0$ along with their standard deviations, and the black line marks $k_f = k_0$ corresponding to ballistic motion.
    }
    \label{fig:single}
\end{figure}

The simulations are performed in three steps to as accurately as possible model a realistic experimental sequence. 
First, we initialize a ground state BEC in a harmonic trap using imaginary time evolution~\cite{chiofalo2000ground}, giving the density distribution plotted as the filled red curve in Fig.~\ref{fig:single}(b)], and follow with real-time evolution.
Second, because the BEC's narrow momentum distribution is centered at $k=0$, we briefly apply a linear potential $\alpha x$ with time-evolution approximately described by the phase factor $\exp(i k_0 x)$, a momentum translation operator that transforms $\ket{k=0}$ to $\ket{k_0}$. 
Third, having prepared our $\ket{k_0}$initial state, we replace the harmonic potential with a speckle potential (with $k_c/\kr = 6$) and follow the time evolution for $16\ {\rm ms}$. Fig.~\ref{fig:single}(a) shows a representative disorder potential that we use in the simulations.
Fig.~\ref{fig:single}(b) captures the main result of this section: when $k_0 > k_c / 2$ the time-evolution is almost unchanged by the speckle potential, while slowly moving initial states are both decelerated and exhibit considerable interference.

Figure~\ref{fig:single}(c) plots the ensemble-averaged momentum $\langle k(t)\rangle$ as a function time for a range of initial states with $k_0$ from near-zero to $k_0 / \kr = 3.3$, and Fig.~\ref{fig:single}(d) plots the final momentum $k_f$ as a function of $k_0$.
At $t=0$, the average momentum is $\langle k(t)\rangle = k_0$; for $k_0\gtrsim k_c/2$ the BEC evolves ballistically, leaving $\langle k(t)\rangle$ unchanged, while $\langle k(t)\rangle = k_0$ falls rapidly for smaller $k_0$.
Both of these observations are consistent with our FGR analysis which showed a complete absence of momentum-changing backscattering for $k_0 \geq k_c/2$, and with rapidly increasing backscattering as $k$ falls to zero.