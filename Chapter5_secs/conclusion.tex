\section{Conclusion} \label{Conclusion}

Our analytical and numerical studies of the transport of SOBECs in disorder potentials clearly show dramatically enhanced transport for initial states in the SOC energy gap.  
The enhanced transport described here results from the same physics giving rise to a spin transistor in Ref.~\cite{Mossman2019}, which also relied on a combination of kinematic and matrix-element effects to yield non-reciprocal appearing transport behavior.
In Sec.~\ref{optical_design}, we describe an explicit experimental proposal using laser speckle derived from $532\ {\rm nm}$ green laser and an off-the-shelf optical diffuser.
In this proposal, SOC is generated from a pair of $790\ {\rm nm}$ laser beams intersecting at the atoms, and initial states would be prepared as described above.
The protection from backscattering is independent of quantum statistics: non-interacting Fermions would experience a conductivity increased by the factor predicted by the FGR when the Fermi energy resides in the SOC gap.
As with the interacting SOBEC we analyzed, we expect that Fermionic systems with moderate interactions would show gains in conductivity, however, the details of this latter case would necessitate future study.

Reference~\cite{Hugel2014} showed that in lattices, the type of 1D SOC in Eq.~\eqref{eq:soc} has the same dispersion as the edge modes of 2D ${\rm Z}_2$ topological insulators~\cite{Kane2005}.
Together with our finding, this indicates that 1D nanowires with SOC either of the Rashba~\cite{Bychkov1984} or linear-Dresselhaus~\cite{dresselhaus1955spin} type should provide the same protection to backscattering from spin-independent disorder as would be observed at the edge of a topological insulator.
