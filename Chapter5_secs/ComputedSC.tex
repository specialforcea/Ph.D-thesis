\subsection{Computed scattering rates}

\begin{figure}[tbp]
    \centering
    \includegraphics{Chapter5_secs/Fig4_Scattering_Rate.pdf}
    \caption{Fermi’s golden rule scattering rate for $\Or /\Er$ = 0, 0.5, 2.5 and 4.0. Left column shows SOC dispersion relations computed for each $\Or$, colored as in Fig.~(\ref{fig:Dispersion relations}). Right column shows normalized scattering rate as a function of initial energy for the initial state $\ket{q_0, -}$ with $q_0 \geq q_{\rm min}$, i.e., in the bottom dispersion and to the right of the higher momentum local energy minimum. The backscattering rate is plotted in black and the forward scattering plotted in red.} %For small $\Or$, the range of $\ket{q_0,\downarrow'}$s in GPE simulation covers two peaks where first order scattering is strong, the top of SOC lower band and bottom of SOC upper band. Between the two peaks is the SOC gap where first order scattering is nearly zero. At $\Or=2.5\Er$, the left peak starts to fall out of the simulation range and when $\Or \geq 4.0\Er$, the left peak disappears. Right panel are the corresponding SOC band structures. }
    \label{fig:scatteringRate}
\end{figure}

We now use these FGR expressions to compute the scattering rates for both forward scattering and back scattering processes.  Because we are interested in transport properties, we define forward scattering processes as those that leave the sign of the group velocity unchanged and back scattering processes and those that do reverse the direction of motion.
We therefore consider initial states $\ket{q_0,-}$ in the lower band with positive group velocity.  Because the lower energy SOC dispersion plotted in Fig.~\ref{fig:scatteringRate} can have a pair of minima located at $\pm q_{\rm min}$, we always select $q_0 > q_{\rm min}$ to assure positive group velocity.  We numerically evaluated the FGR for $^{87}{\rm Rb}$ atoms illuminated with $\lambdar=790\ {\rm nm}$ Raman lasers, giving $\Er = h \times 3.7\ {\rm kHz}$, and for speckle with $k_c = 6 \kr$.  The $t=13.4\ {\rm ms}$ interaction time was selected to be experimentally relevant.  

The right panels of Fig.~\ref{fig:scatteringRate} show the normalized scattering rate computed for four different values of $\Or$, with the back-scattering rate plotted in back and forward scattering plotted in red.  These rates combine the contributions from the $\pm$ bands in Eq.~(\ref{transfer prob}).

Fig.~\ref{fig:scatteringRate}(a), computed for $\Or=0$ (equivalent to the case with no SOC), shows two key effects.  First, the diverging forward- and back-scattering rates at low energy follow from the diverging density of states (DoS) in 1D.  Second, as expected, the rate of backscattering (black) falls to zero when $\delta q > k_c$, while forward scattering (red) simply falls with the DoS.

Fig.~\ref{fig:scatteringRate}(b) and Fig.~\ref{fig:scatteringRate}(c) show cases with a well resolved SOC energy gap.  As expected, backscattering is nearly completely suppressed for initial energies in the energy gap, while forward scattering is hardly changed.  In addition, a pair of singular features border of the energy gap, resulting from the diverging DoS and the local extrema of the dispersions.  Fig.~\ref{fig:scatteringRate}(d) shows the same phenomena, but just as the two minima at $\pm q_{\rm min}$ have merged into a single minimum at $q_{\rm min}=0$.

We therefore conclude, for non-interacting particles back scattering and momentum relaxation is nearly completely suppressed for atoms starting in the SOC energy gap.