\subsection{Gross-Pitaevskii equations}
Here we numerically study the deceleration of an SOBEC initially moving in a speckle potential using the time-dependent Gross-Pitaevskii equation (GPE).
The time-dependent GPE 
\begin{align}\label{gpe}
i\hbar\partial_t\Psi({\bf r},t) &= \left[-\frac{\hbar^2}{2m}\nabla^2 + V({\bf r}) + g_{\rm 3D}|\Psi({\bf r},t)|^2\right]\Psi({\bf r},t)
\end{align}
is a non-perturbative dynamical description ~\cite{erdHos2010derivation} of a large number of interacting identical bosons occupying the same spatial mode $\Psi({\bf r}, t)$, normalized to the total atom number, $N=\int d^4{\bf r} |\Psi({\bf r}, t)|^2$
The interaction strength $g_{\rm 3D} = 4\pi\hbar^2a_s/m$ can be expressed in terms of the $s$-wave scattering length $a_s$.
This GPE provides a good description of low-temperature spin-polarized BECs, with negligible thermal excitations~\cite{Dalfovo1999}. 

Since our focus is on 1D transport, we must first obtain a 1D description of our 3D system~\cite{bao2003numerical}.
Here we, we assume that the potential $V({\bf r}) = V(x) + V_\perp(y,z)$ can be separated into a weak longitudinal potential $V_\parallel(x)$ along with a strongly confining transverse potential $V_\perp(y,z)$.
When the single-particle energy spacing from $V_\perp(y,z)$ greatly exceeds all other energy scales, the 3D wavefunction can be factorized into
\begin{equation}
    \Psi({\bf r},t) = \psi(x,t)\phi(y, z),
\end{equation}
containing a longitudinal term of interest giving the 1D density $n(x) = |\psi(x,t)|^2$, and a transverse term $\phi(y, z)$, normalized to unity, assumed to be the ground state of the transverse potential. 
Inserting this ansatz into the GPE and integrating out the transverse degrees of freedom, gives the 1D GPE
\begin{equation}
    i\hbar\partial_t\psi = \left[-\frac{\hbar^2}{2m}\partial_x^2 + V(x) + g|\psi|^2\right]\psi,
\end{equation}
suitable for studying single-component 1D bosons in a speckle potential with 1D interaction strength 
\begin{equation}
    g = g_{\rm 3D}\int dy dz |\phi(y,z)|^4.
\end{equation}
For compactness of notation, here and below, we shall omit the functional dependance of $\psi$ on $x$ and $t$.


The two component 1D spinor GPE describing SOBECs extends Eq.~\eqref{eq:soc} to include interactions, and consists of a pair of coupled non-linear differential equations
\begin{align}
i\hbar\partial_t\psi_\uparrow &= \left[\frac{\hbar^2}{2m}\left(-i\partial_x + \kr\right)^2 + \frac{\delta}{2} + V(x) + g_{\uparrow\uparrow} |\psi_\uparrow|^2 + g_{\uparrow\downarrow}|\psi_\downarrow|^2\right]\psi_\uparrow + \frac{\Or}{2}\psi_\downarrow \\
i\hbar\partial_t\psi_\downarrow &= \left[\frac{\hbar^2}{2m}\left(-i\partial_x - \kr\right)^2 - \frac{\delta}{2} + V(x) + g_{\downarrow\downarrow} |\psi_\downarrow|^2 + g_{\uparrow\downarrow}|\psi_\uparrow|^2\right]\psi_\downarrow + \frac{\Or}{2}\psi_\uparrow
\end{align}
including the interaction strengths $g_{\uparrow\uparrow}$, $g_{\uparrow\downarrow}$, and $g_{\downarrow\downarrow}$.
Here we focus on the specific case of $^{87}{\rm Rb}$ atoms~\cite{kawaguchi2012spinor} in the $f=1$ ground state manifold and have selected $\ket{\uparrow} = \ket{m_F=0}$ and $\ket{\downarrow} = \ket{m_F = -1}$. 
The interactions can be parameterized in terms of an $s$-wave pseudo-potential $(g_{0, {\rm 3D}} + g_{2, {\rm 3D}}\Vec{F}_\alpha\cdot \Vec{F}_\beta) \delta({\bf r}_i - {\bf r}_j)$ now dependent on spin.
In $^{87}{\rm Rb}$'s $f = 1$ manifold $g_{0, {\rm 3D}} = 100.86 \times 4\pi\hbar^2 a_{\rm B}/m$ is vastly larger than $g_{2, {\rm 3D}}\approx-4.7\times10^{-3} \times g_{0, {\rm 3D}}$, where $a_{\rm B}$ is the Bohr radius~\cite{vanKempen2002,Widera2006}.
The interaction coefficients reduce to effective 1D interaction strengths just as in the single component case, and are related to the generic coefficients~\cite{Ho1998,Ohmi1998} via $g_{\uparrow\uparrow} = g_0$ and $g_{\downarrow\downarrow}  = g_{\uparrow\downarrow} = g_0 + g_2$.
Table \ref{table:1} summarizes the parameters used in our simulations. 


Our simulation results are divided into two sections: Sec.~\ref{single} hones our understanding by considering a single-component BEC evolving in a speckle potential, and then in Sec.~\ref{soc} we contrast to the case with SOC. 
In both sections, we simulate initially trapped BECs accelerated to an initial momentum $k_0$ or quasi-momentum $q_0$ and we study their deceleration. 
All the results are averaged over 20 speckle realizations, as in  Fig.~\ref{fig:Optical speckle}(b). 
The average speckle potential $h\times 200\ {\rm Hz} \approx 0.05\Er$ was selected to be weak enough to cause no trapping effect yet strong enough to produce significant deceleration within $15 {\rm ms}$.

\begin{table}
\renewcommand{\arraystretch}{1.25}
\begin{tabular}{l c c}
\hline\hline
Description & Symbol & Value \\
\hline
%interaction coefficient & $g_{0, {\rm 3D}}$ & $8.7\times10^{-52}$~${\rm J} \cdot {\rm m}$\\
%interaction coefficient & $g_{2, {\rm 3D}}$ & $-4.0\times10^{-54}$~${\rm J} \cdot {\rm m}$\\
$^{87}{\rm Rb}$ atomic mass & $m$ & $1.42\times10^{-25}~{\rm kg}$\\
Raman laser wavelength & $\lambdar$ & $790~{\rm nm}$ \\
Recoil energy & $\Er$ & $h\times 3.678\ {\rm kHz}$\\
Dipole trap frequency & $\omega/2\pi$ & $10~{\rm Hz}$  \\
Angle of Raman beams & $\theta_{\rm R}$& $180^{\circ}$ \\
Speckle potential cut off &  $k_c$ & $6\kr$\\
Average speckle potential & $\overline{V(x)}$ & $0.05\Er$\\
Grid spacing  & $\delta x$ & $66\ {\rm nm}$\\
Grids points (single-component)& $N_x$ & $2^{14} + 1$\\
Grids points (SOC)& $N_x$ & $2^{13} + 1$\\
Atom number & $N$ & $2\times 10^5$ \\
Chemical potential & $\mu$ & $h\times 300\ {\rm Hz}$ \\
% 300Hz, TF radius 34um.
\hline\hline
\end{tabular}
\normalsize
\caption{Simulation parameters}
\label{table:1}
\end{table}

