\section{Scattering of an SOBEC from a speckle potential} \label{Model}

\begin{figure}[htbp]
    \centering
    \includegraphics{Chapter5_secs/Fig3_Dispersions.pdf}
    \caption{Fermi's Golden Rule. Momentum are expressed in units of the single-photon recoil momentum $\kr$ used to create SOC in (c). (a) Representative PSD for optical speckle with $k_c = 6 \kr$. (b) Free particle dispersion relation. The dashed arrow marks the boundary above which the FGR rate vanishes, while the solid arrow provide an example with non-zero rate. (c) SOC dispersion relations computed for $\delta = 0$ add $\Or = 1\Er$ colored according to the expectation value $\langle\sigma_z(q)\rangle$, with arrows marked as in (b).  Note the transition through the gap in the dispersion relation at $E\approx \Er$ where the FGR rate is nearly zero.}
    \label{fig:Dispersion relations}
\end{figure}

We now focus on the motion of spin-orbit coupled bosons in a speckle-induced disorder potential.  In this section, we develop a Fermi's golden rule (FGR) approach for scattering from a disorder potential, both with and without SOC, schematically depicted in Fig.~\ref{fig:Dispersion relations}.   The first order scattering processes captured by the FGR are possible when a matrix element (here from the disorder potential) can couple energetically degenerate initial and final states (here momentum or quasi-momentum states).  We will see that the strength of this coupling is proportional to the PSD of the speckle potential, an example of which is shown in Fig.~\ref{fig:Dispersion relations}(a).  As depicted in  Fig.~\ref{fig:Dispersion relations}(b), this implies an absence of scattering for momenta differences larger than the speckle-cutoff $k_c$.  Adding SOC, as in Fig.~\ref{fig:Dispersion relations}(c), can suppress scattering for additional wavevectors.  Because a spin-independent speckle potential has no spin-changing matrix element, the energetically allowed transition at an energy $E/\Er \approx 1$ between states of opposite spin is strongly suppressed.  The following discussion quantifies these observations.

\subsection{Spinless atoms}\label{sec_spinless_atoms}

For spinless free particles, the unperturbed Hamiltonian $H = \hbar^2 k^2 / 2m$ implies that we study scattering between initial and final momentum states, labeled by  $\ket{k_0}$ and $\ket{k_f}$ respectively.  Figure~\ref{fig:Dispersion relations}(b) depicts examples by open circles, with arrows connecting initial states to final states.

The time evolution of the initial state $\ket{\psi(0)} = \ket{k_0}$ subject to the speckle potential $V(x)$ may always be expressed as
\begin{align} 
\ket{\psi(t)} &= \sum_{k}C_{k,k_0} (t) e^{-i\omega_k t}\ket{k}, \label{expand}
\end{align}
with $C_{k,k_0}(0) = \delta_{k,k_0}$ and $\hbar\omega_k = \hbar^2 k^2 / 2 m$.  The coefficients $C_{k,k_0} (t)$ are governed by the time-dependent Schr\"{o}dinger equation giving the exact expression
\begin{align}
    C_{k_f,k_0}(t) = &  C_{k_f,k_0}(0) + \label{allorder} \\
    & \frac{1}{i\hbar}\sum_k \bra{k_f}\hat V\ket{k} \int_0^t d\tau e^{i\omega_{k_f,k}\tau}C_{k,k_0}(\tau).\nonumber
\end{align}
with 
\begin{align}
    \omega_{k,l} &= \omega_k-\omega_l, &{\rm and} && \hat V &= \sum_x V(x) \ket{x}\bra{x}.
\end{align}
An order-by-order perturbation theory is typically obtained by recursively inserting the integral expression for $C_{k_f,k_0}(t)$ back into the integrand; unfortunately, the general problem is intractable and we truncate the perturbation series at first order. This term is effectivly obtained by replacing $C_{k_f,k_0}(\tau)$ with $C_{k_f,k_0}(0) = \delta_{k_f,k_0}$, and find
\begin{equation}\label{firstC}
    C_{k_f,k_0} (t) = \delta_{k_f,k_0} + \frac{1}{i\hbar}\int_0^t d\tau\bra{k_f}\hat V\ket{k_0} e^{i\omega_{k_f,k_0}\tau}.
\end{equation}
Unfortunately, we do not know $V(x)$ for any specific realization of the speckle potential.

In Chpt.~\ref{speckle_chapter} we characterized optical speckle in terms of second-order statistical metrics such as the PSD, here equal to $\rho(k_f-k_0) = \langle\bra{k_f}\hat V\ket{k_0}\bra{k_0}\hat V\ket{k_f}\rangle$, where the double-brackets indicate the ensemble average. The resulting ensemble averaged transition probability
\begin{equation}
 P_{f,0} (t) = \frac{\rho(k_f- k_0)}{\hbar^2}\left[\frac{2}{\omega_{f,0}} \sin\left(\frac{\omega_{f,0}t}{2}\right)\right]^2
\end{equation}
is a sharply peaked function centered at $\omega_{f,0} = 0$ with width $2\pi/t$, showing that a narrow range of energy matching states can be populated.  For long times, $\omega_{f,0}t \gg 1$ the quantity in square brackets converges to a scaled Dirac delta function $t\times\delta(\omega_{f,0})$.

Figure~\ref{fig:Dispersion relations}(a) displays the normalized PSD for a speckle potential computed with $k_c = 6 \kr$, reminding us that $\rho(k) = 0$ for $k\geq k_c$.  Our FGR expression allows two types of scattering processes for the free particle dispersion shown in Fig.~\ref{fig:Dispersion relations}(b).  In the first process, depicted by the black arrow, the atom's initial momentum is reversed, changed by $\Delta k = 2 k_0$; as indicated by the dashed line, this process is second-order forbidden for $k_0 \geq k_c/2$.  In the second process (not pictured), the atom's momentum is only infinitesimally changed: spreading the wave-packet, but leaving the average momentum unchanged.
This picture shows that backscattering is essential for momentum-relaxation.
\subsection{Spin-orbit coupled atoms}

Our 1D SOC coupling~\cite{lin2011spin} is created by illuminating a two-level atom with a pair of counter-propagating lasers with wavelength $\lambdar$ tuned to drive stimulated Raman transitions between states $\{ \ket{q+\kr,\uparrow},\ket{q-\kr,\downarrow} \}$.  Here $\hbar\kr = 2\pi\hbar/\lambdar$ and $\Er = \hbar^2 \kr^2 / 2m$ are the single-photon Raman recoil momentum and energy respectively.
Subject to this Raman coupling, the atoms obey the 1D Hamiltonian \begin{align}
\hat{H}(q) = \left[\frac{\hbar^2 q^2}{2m} + \frac{\hbar^2 \kr^2}{2m}\right]\hat 1 + \left(\frac{\hbar^2 \kr q}{m} + \frac{\delta}{2}\right)\hat \sigma_z + \frac{\hbar\Or}{2}\hat \sigma_x,\label{eq:soc}
\end{align}
where $\left\{\hat 1, \hat\sigma_x,\hat\sigma_y,\hat\sigma_z\right\}$ are the identity and Pauli operators, respectively.
Here $q$ is the quasi-momentum, $\Or$ is Raman coupling strength, and $\delta$ is the detuning from the two-photon Raman resonance condition.
The resulting dispersion relations, plotted in Fig.~\ref{fig:Dispersion relations} for $\delta = 0$ and $\Or = \Er$, have energies $E^\pm(q)$  labeled by $q$ along with $\pm$ to indicate if they are in the upper or lower band.

These new energies and their associated amplitudes
\begin{align*}
\ket{q,\pm} &\propto a_\pm(q)\ket{q-\kr,\downarrow} + b_\pm(q)\ket{q+\kr,\uparrow}
\end{align*}
change the potential scattering processes, which we again compute using a FGR expression.  The coefficients
\begin{align*}
a_\pm(q) & = \pm\frac{\Or}{2}& {\rm and} && b_\pm(q) &= \pm\frac{\Delta(q)}{2} + \frac{\sqrt{\Delta^2(q) + \Or^2}}{2}.
\end{align*}
along with the quasi momentum dependent detuning
\begin{align}\label{delta}
    \Delta(q) &= \frac{2\hbar^2q\kr}{m} + \delta
\end{align}
fully define these superposition states.

Following the same FGR argument presented above for initial states $\ket{q_0,-}$ in the lower dispersion scattering from a spin-independent speckle potential, we find scattering probabilities
\begin{align}\label{transfer prob}
    P^\pm_{f, 0}(t) &= \frac{\rho(\Delta q)}{\hbar^2}\left|\frac{2\sin(\omega^\pm_{f, 0}t)}{\omega^\pm_{f, 0}}\!\matrixel{q_f,\pm}{e^{i\Delta q x}}{q_0,-}\right|^2
\end{align}
expressed in terms of the quasimomentum and energy differences $\hbar\Delta q = \hbar q_f-\hbar q_0$ and $\hbar \omega_{f,0}^\pm = E^\pm(q_f) - E^-(q_0)$.  For most initial states $\ket{q_0,-}$, such as two higher higher-energy states marked in Fig.~\ref{fig:Dispersion relations}(c), the scattering is essentially unchanged from our spinless example, with scattering occurring between energy-matched states with the same initial and final spin.  In contrast, for initial states residing in the SOC energy gap there is no energy-matched state of the same spin available for back-scattering; as indicated by the dashed line scattering is greatly suppressed.  We note that that backscattering is not completely blocked, because the energy matching states $\ket{\pm q_0,-}$ are not spin-eigenstates and do have some spin-overlap.
\subsection{Computed scattering rates}

\begin{figure}[tbp]
    \centering
    \includegraphics{Chapter5_secs/Fig4_Scattering_Rate.pdf}
    \caption{First order average scattering Rate. The right panel is the normalized average scattering rate for the initial 13.4ms, for each initial state $\ket{q_0,\downarrow'}$ on the right side of the bottom band. The left panel shows the corresponding dispersion relation, for coupling strength $\Or = 0, 0.5 \Er, 2.5\Er$ and $ 4.0\Er$. } %For small $\Or$, the range of $\ket{q_0,\downarrow'}$s in GPE simulation covers two peaks where first order scattering is strong, the top of SOC lower band and bottom of SOC upper band. Between the two peaks is the SOC gap where first order scattering is nearly zero. At $\Or=2.5\Er$, the left peak starts to fall out of the simulation range and when $\Or \geq 4.0\Er$, the left peak disappears. Right panel are the corresponding SOC band structures. }
    \label{fig:scatteringRate}
\end{figure}

We now use these FGR expressions to compute the scattering rates for both forward scattering and back scattering processes.  Because we are interested in transport properties, we define forward scattering processes as those that leave the sign of the group velocity unchanged and back scattering processes and those that do reverse the direction of motion.
We therefore consider initial states $\ket{q_0,-}$ in the lower band with positive group velocity.  Because the lower energy SOC dispersion plotted in Fig.~\ref{fig:scatteringRate} can have a pair of minima located at $\pm q_{\rm min}$, we always select $q_0 > q_{\rm min}$ to assure positive group velocity.  We numerically evaluated the FGR for $^{87}{\rm Rb}$ atoms illuminated with $\lambdar=790\ {\rm nm}$ Raman lasers, giving $\Er = h \times 3.7\ {\rm kHz}$, and for speckle with $k_c = 6 \kr$.  The $t=13.4\ {\rm ms}$ interaction time was selected to be experimentally relevant.  

The right panels of Fig.~\ref{fig:scatteringRate} show the normalized scattering rate computed for four different values of $\Or$, with the back-scattering rate plotted in back and forward scattering plotted in gray.  These rates combine the contributions from the $\pm$ bands in Eq.~(\ref{transfer prob}).

Panel (a), computed for $\Or=0$ (equivalent to the case with no SOC), shows two key effects.  Firstly, the diverging forward and back scattering rates at low energy follow from the diverging density of states (DoS) in 1D.  Secondly, as expected, the rate of back scattering falls to zero when $\delta q > k_c$.

Panels (b) and (c) show cases with a well resolved SOC energy gap.  As we hoped for, back scattering is nearly completely suppressed for initial energies in the energy gap, while forward scattering is hardly changed.  In addition a pair of singular features boarder the energy gap, resulting from the diverging DoS the the local extrema of the dispersions.  Panel (d) shows the same phenomena, but just as the two minima at $\pm q_{\rm min}$ have merged into a single minimum at $q_{\rm min}=0$.

We therefore conclude, for non-interacting particles back scattering and momentum relaxation is nearly completely suppressed for atoms starting in the SOC energy gap.