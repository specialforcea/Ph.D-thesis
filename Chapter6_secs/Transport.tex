\section{Transport of spinless BECs in speckle potentials}\label{transport}

In chapter (\ref{chpt 5}), we study the transport of spinless BECs under speckle potentials. As Fig.~ (\ref{fig:single}) shows, a BEC with chemical potential $~300 {\rm Hz}$ travel through speckle potentials with average potential depth $~200 {\rm Hz}$ could be scattered by the speckle potential and decelerate. The deceleration of a BEC depends on its initial velocity and the cutoff $k_c$ in the speckle potential PSD. As Fig.~ \ref{fig:single}(d) shows, after evolving under the speckle potential for $16 {\rm mm}$, the BECs with small initial velocity has more significant deceleration. For BECs with large initial velocity, $k_0>k_c/2$, the deceleration is minimal during the $16 {\rm mm}$.

In experiment, we would like to see how BECs with different velocities decelerate. As discussed in chapter (\ref{speckle_chapter}), we can make speckle potentials that have the same PSD as our numerical simulation. And as discussed in Sec. ~(\ref{speckle_pulsing}), the average speckle potential depth can be measured. So an ideal experimental sequence is to have the BEC travel with constant velocity under well calibrated speckle potential, and the final velocity would be measured by using in-situ or TOF absorption images. To that end, the first challenge we are faced with is that how to make a BEC travel with a constant velocity for an extensive amount of time ($16 {\rm mm}$ in the simulation). We make BECs by doing evaporative cooling in a crossed dipole trap as discussed in Sec.~(\ref{dipole trap}), so our first choice is to make BECs travel in the crossed dipole trap. As Eq.~(\ref{dipole_poten}) suggests, dipole potential is proportional to the intensity of the dipole beam. 

In our case, we image atoms in z direction and measure the motion of atoms in x direction. The dipole potential in x direction is a combination of the dipole potential from z dipole beam and x dipole beam. The width of the dipole potential from x dipole beam is the Rayleigh length, which in our case is $~ 1.3{\rm mm}$. Based on our design, the velocity of atoms corresponds to the recoil $k$ vector $k_r$ is $3.3 {\rm \mu m/ms}$. We expect the atoms to move less than $50 {\rm \mu m}$ during the experiment, so the dipole potential from x dipole beam can be ignored.

In x direction, the dipole potential from the z dipole beam is
\begin{equation}
    V_{dip}(x) = -V_0\exp{-\frac{2x^2}{w^2}}.
\end{equation}
where $w$ is the width of the beam at the atoms. Expand the potential at $x=0$ to second order,
\begin{equation}
    V_{dip}(x) \approx -V_0 + \frac{2V_0}{w^2}x^2,
\end{equation}
has a quadratic form. Around the center of the trap, the dipole potential can be approximated by a harmonic potential with frequency $\sqrt{\frac{4V_0}{w^2}}$.

In the ideal case, the atoms would move at a constant velocity in the dipole trap, meaning the frequency $\sqrt{\frac{4V_0}{w^2}}$ is zero and the z dipole beam is completely turned off. More realistically, if the velocity of the atoms change by less than 5\% at the center of the trap in $15 {\rm ms}$, the period of the harmonic oscillation needs to be more than $300 {\rm ms}$. So the trapping frequency is around $3 {\rm Hz}$. 

The x dipole beam and the z dipole beam in our experiment are the first order and zeroth order beams from an AOM, the total power of the two beams are conserved. We optimized the ratio of power of the two beams to maximize the phase space density of the BECs after the dipole evaporation stage. 

