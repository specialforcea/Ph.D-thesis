\section{Correlation length}
The field-field correlation length
\begin{align}
    c_E(z)^2 &= \frac{\iint |C_E({\bf r}_1,{\bf r}_2; z)||{\bf r}_1-{\bf r}_2|^2d^2{\bf r}_1d^2{\bf r}_2}{\iint |C_E({\bf r}_1,{\bf r}_2; z)|d^2{\bf r}_1d^2{\bf r}_2}\\
    &= \frac{2 w(z)^2 \sigma(z)^2}{2w(z)^2 + \sigma(z)^2} \approx \sigma(z)^2
\end{align}
obtained from Eq.~(\ref{eq:C_E}), sets the scale over which the electric field retains its spatial coherence.  The field-field correlation length is minimized at $z=z_0$, and is always larger than $\sigma$.  Generally, speckle beams operate in the regime $w \gg \sigma$, where there are many speckle grains within a large beam, giving the final approximate relation. 

As was already noted in our ray-optics discussion, this has important implications for experiment design.  For cold atom experiments such as ours, the large momentum-change imparted by short-length scale speckle is essential, where a correlation length at or below the micron scale is desirable.  Since the correlation length available for typical commercial diffusers ranges from $10\ {\rm \mu m}$ to $100\ {\rm \mu m}$, an additional focusing stage is required.  

A focusing lens can easily take the $10\ {\rm \mu m}$ to $100\ {\rm \mu m}$ correlation length available for typical commercial diffusers and create a beam with sub-micrometer correlation length at its focus.  Fig.~\ref{fig:speckle1}(c) compares the correlation length of a beam with (red solid) and without (red dashed) a focusing lens for the specific case of an initial laser beam of wavelength $\lambda = 532\ {\rm nm}$ with Gaussian beam parameters: focal point $z_0=0$, beam waist $w = 25\ {\rm mm}$ and correlation length $\sigma = 100\ \mu{\rm m}$.  This beam is focused by a lens of focal length $f=100\ {\rm mm}$, the correlation length at the focus is $c_E=0.96\ \mu{\rm m}$.  The remaining derived beam parameters are
$M^2 \approx 1.25\times10^5$, $z_R \approx 3.7\ {\rm km}$, and $z_R^* \approx 10.4\ {\rm m}$. 

In \cite{yura1999three}, the space–time evolution of three-dimensional (3D) optical speckle is studied using the ABCD ray-matrix techniques. The optical speckle they studied results from a diffuse object that is illuminated by a Gaussian-shaped laser beam. The field-field correlation length obtained from this approach agrees with our results. In addition, the intensity-intensity correlation length in $z$ direction is calculated in \cite{yura1999three}. In the case the optical speckle is focused by a lens with focal length $f$, the on-axis intensity-intensity correlation length in $z$ direction $L_z$ is of the order of the depth of focus $4f^2/kw^2$. Using the parameters in Fig.~\ref{fig:speckle1}(c), in the focal plane, $L_z \approx 5.4~{\rm \mu m}$. In the free propagation case, $L_z$ grows as $4z^2/kw^2$, and $L_z \gg c_E(z)$ in the far field where $z \gg w$.
