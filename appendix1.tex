\appendix
\section{Postulates of Quantum Mechanics}
\label{appendix:QMP}
1.At each instant the state of a physical system is represented by $\ket{\psi}$ in the space of
states.

2.Every observable attribute of a physical system is described by an operator that acts on the
kets that describe the system.

3.The only possible result of the measurement of an observable $A$ is one of the eigenvalues of
the corresponding operator $\hat{\mathcal{A}}$.

4.When a measurement of an observable $\mathcal{A}$ is made on a generic state $\ket{\psi}$, the probability
of obtaining an eigenvalue an is given by the square of the inner product of $\ket{\psi}$ with the
eigenstate $\ket{a_n}$, $|\bra{a_n}\ket{\psi}|^2$.

5.Immediately after the measurement of an observable $\mathcal{A}$ has yielded a value $a_n$, the state of
the system is the normalized eigenstate $\ket{a_n}$.

6.The time evolution of a quantum system preserves the normalization of the associated ket.
The time evolution of the state of a quantum system is described by 
\begin{equation}
    i\hbar\frac{d}{dt}\ket{\psi(t)} = \mathcal{H}\ket{\psi(t)}
\end{equation}


